\subsection{Preprocessing on Query Graph}

\subsubsection{Constraint Tightening}
Oftentimes a query graph will have contradictory or superfluous temporal
information, for example, the local interval of a given edge may not intersect
at all with the temporal condition or the local intervals given will preclude
the explicit semantics, in which case, we can reject the query instantly with a
null result.

Another interesting situation is when, for a given query $<Q, T_q,
\intersection, \concurrent>$ and some edge $e \in E(G)$ with $T(e) \neq T_q \cap
T(e)$ and $T(e) \cap T_q \neq \emptyset$. Then, if we enforce the
\intersection{} semantics, we do not need to consider, for the edge $e$, the
part of the interval $T(e) - T_q$, so we can update $T(e)$ to be $T(e) \cap
T_q$.

Further, we need to consider the way that the \concurrent{} semantics will
propogate to the neighbors of $e$. For this case, we need to enforce the
\intersection{} semantics for every edge in $Q$, meaning that we can rewrite
each $T(e)$ to be $T_q \cap \bigcap_{e' \in E} T(e')$. Of course for the
$\exact$ semantics this loses pruning power, since we can just perform an index
search on the edges of the data-graph to find all time-restricted candidates. We
can perform a similar tightening for almost every combination in $\impTC \times
\expTC$. Here are the ones for which a tightening makes sense, note that after
transformation, the explicit and implicit semantics remain the same.

\begin{center}
  \begin{tabular}{cc c c} \toprule
    $\impTC$      & $\expTC$   & start & end \\ \midrule
    \concurrent   & \contain, \intersect
      & $\min \left(\bigcap_{e' \in N(e)} T(e')\right)$ 
      & $\max \left(\bigcap_{e' \in N(e)} T(e') \right)$ \\
    \strongConsec & \contain, \intersection
      & $\min \left(\bigcup_{e' \in pred(e)} T(e')\right)$
      & $\max \left(\bigcup_{e' \in succ(e)} T(e')\right)$ \\
    \strongConsec & \contained
      & $\max \left(\bigcup_{e' \in pred(e)} T(e')\right)$
      & $\min \left(\bigcup_{e' \in succ(e)} T(e')\right)$ \\
    \bottomrule
  \end{tabular}
\end{center}

The above rewriting rules are only improvements on less restrictive intervals
and will not always be an improvement on the given rules (except in the case of
the \concurrent{} semantics. 

\todo[inline]{We need proofs of these.  In an actual paper these should go in an
Appendix, since for the purpose of the paper we don't care why they exist, just
that there are such rewrite rules.}

\subsubsection{Hypergraph Compression}

We can also identify substructures of the query graph for which the semantic
conditions are equivalent. What? that's possible? tell me more... Consider the
following example of a cycle. Its a well-known fact that interval graphs that
contain a cycle are chordal, so for 2-cycles all three semantics are equivalent,
and for 3-cycles the \concurrent{} and \strongConsec{} semantics are
equivalent.

We can also note that for a star in which the in- and out-degrees of the central
vertex are at least one, the \concurrent{} and \strongConsec{} semantics
are equivalent.

So, given a specific query, we can decompose it maximually into 2-cycles, and
3-cycles (i.e. if a 3-cycle contains a 2-cycle add the 3-cycle instead of the
2-cycle) and enforce the temporal constraints enforced by the tight
semantics. Note that these are not the only structures for which the semantics
are the same, but they were the easiest to detect.

\begin{conjecture}
  3-cycles are the largest structures (w.r.t. number of vertices) that allow us to
  reduce \strongConsec{} semantics to \concurrent{}.
\end{conjecture}

\begin{proof}
  \todo[inline]{fill in here}.
\end{proof}

So, we will detect such two and three cycles using an $O(|V_Q||E_Q|)$ method,
since we are assuming small query graphs with no more than 20 or 30 edges, we
can store this in memory. The detection algorithm is defined in
\ref{alg:detect_substructs}. It only makes sense to run this algorithm for
$\weakConsec$ and $\strongConsec$ since we are attempting to leverage the
selectivity of the $\concurrent$ semantics.

\begin{algorithm}
  \label{alg:detect_substructs}
  \caption{\textsc{DetectSubstructs}}
  \KwIn{A query graph $Q$ and an implicit semantic $m$}
  \KwOut{A set of structures reducing to $\concurrent$}
  \SetAlgoLined

  Let the set of cycles $C$ \texttt{:=} $\emptyset$\;
  \ForEach{$e = (u,v) \in E(Q)$}{
    \ForEach{$e' = (v,u) \in E(Q)$}{
      add $\{e,e'\}$ to $T$\;
    }
    // Since $\weakConsec$ does not rewrite well \;i
    \If{$\impVar \neq \weakConsec$}{
      \ForEach{$w \in V(Q)$}{
        \ForEach{ pair of edges $e' = (v, w), e'' = (w,u)$}{
          add $(e,e', e'')$ to $T$\;
        }
      }
    }
  }
\end{algorithm}

Then, for each of these substructures $S_e$, we will build a hypernode, given $m
\in \impTC$, and $x \in \expTC$, $N_{S_e}^{m}$ that reduces to $\concurrent$
semantics. This hypernode will then have an active window itself.

We will then define a hypernode profile, which is an extended version of the
temporal profile (see section \ref{sec:encoding}). For a given node $n$, it is a
tuple of the lexographically ordered labels ($p_s$),
$\impApprox(\concurrent)(S_e)$, and $\impApprox(\impVar)(S_e)$. When we traverse the
tree to try and map these vertices, we will need to prune results such that the
internal edges obey the $\intersection$ explicit semantics with
$\impApprox(\concurrent)(S_e)$, and the incoming and outgoing edges obey the $m$
implicit semantics with respect to $\impApprox(\impVar)(S_e)$. Algorithm
\ref{alg:hypernode_matching} describes the hypernode matching process.

\begin{algorithm}
  \label{alg:hypernode_matching}
  \caption{HypernodeMatching}

  \KwIn{A query hypernode $n$ and its profile $(p_s, p_{in}, p_{out})$, a
    data hypernode $n'$ and its profile $(p'_s, p'_{in}, p'_{out})$, the
    set of candidate sets $\candSet$}

  \KwOut{The updated candidates sets $\candSet$}
  \SetAlgoLined
  

  \If{$p_s \neq p'_s$ or $p_{in} \cap p'_{in} = \emptyset)$ or $|E(n)| > |E(n')|$}{
    \ForEach{$e \in E(n)$}{
      remove $E(n')$ from $\candSet(e)$\;
    }
  }
\end{algorithm}

Once we have constructed such hypernodes, we can re-tighten the constraints on
the graph as per the rules defined above.

Further, when traversing the graph, we will enforce, for $f$ an incoming or
outgoing edge to a matched hypernode in the data graph, $\impApprox(\impVar)\{T(e),
p_{out}\}$.
