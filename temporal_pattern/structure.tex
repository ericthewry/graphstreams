\section{Implementation Structure}
\label{sec:structure}

Since Ullman's original search-space pruning algorithm~\cite{1976-ACMJ-Ullman}
published in 1976 there has been an influx of new algorithms attempting to find
tighter subspaces and improved search orders, as well as storing partial results
in graph indexes to allow for faster access.  The most recent, and fastest
algorithms have been in the last 8 years. Notably, these are
QuickSI~\cite{2008-VLDB-QuickSI}, GraphQL~\cite{2008-SIGMOD-GraphQL},
TurboIso~\cite{2013-SIGMOD-TurboISO}, BoostIso~\cite{2015-VLDB-BoostIso}, and
DualIso~\cite{2014-IEEE-DualIso}.

A 2012 comparison of existing algorithms~\cite{2012-VLDB-IsoSurvey} concluded
that QuickSI, and GraphQL were the fastest from among other algorithms including
GADDI~\cite{2009-EDBT-GADDI}, SPath~\cite{2010-VLDB-SPath}, and
VF2~\cite{2004-PAMI-VF2}. It created a common framework for all of the
algorithms, that allowed for a more comprehensive understanding of the behavior
of these algorithms.

We chose the most performant of these algorithms on large data sets, the
verification algorithm of QuickSI, and extended it simply to allow it to behave
well for temporal graphs. Additionally we extended the algorithm to allow for
multigraph queries since it is rare for a temporal data graph to be purely
simple. A final important change is that our version of QuickSI, which we will
call QuickTSI, is an \textit{edge} isomorphism, instead of the standard vertex
isomorphism.

\subsection{Graph}

In this section we describe the data structures used to prepresent graphs in the
implementation. We use the adjacency list representation, which allows for
constant retrieval of the edges incident to any given edge. We also have an edge
index (\texttt{FeatureIndex}) built into the representation of the graph, which
is currently unused, but can be used for retrieving the number of isomorphism
candidates for an edge.

\subsubsection{Edge and Vertex}
An \texttt{Edge}, and similarly a \texttt{WeightedEdge}, are
structs~\ref{struct:edge} containing an id, pointers to its endpoints, a start
time, an end time, a label, and in the case of a \texttt{WeightedEdge}, a
weight. The types EdgeId, Time, and Label are all just \texttt{int}s, but for
expressivity (and modularity) we use \texttt{typedef}s to rename them. This
helps to clarify the purpose of the variable, as well as allowing easy
replacement with any type that supports comparison and hashing.

\begin{algorithm}[H]
  \label{struct:edge}
  \MyStruct{(Weighted)Edge}{
    EdgeID ident\;
    Vertex* u\;
    Vertex* v\;
    Time start\;
    Time end\;
    Label l\;
    /* Weighted Only */
    int weight\;
  }
\end{algorithm}

This then leads to the question, How do we define a \texttt{Vertex}? Its a
struct containing an  identifier and a label, where the type of the identifier
is VertexID, which is a wrapper for int.

\begin{algorithm}[H]
  \label{struct:vertex}
  \MyStruct{Vertex}{
    VertexID ident\;
    Label l\;
  } 
\end{algorithm}

An annoying side effect of defining these as structs and not classes is that
they are not template-able so for every following definition there will be an
unweighted and a weighted version. This could be solved simply by paramaterizing
on the Edge type, so for the purposes of this document we will define the
unweighted type, with the weighted type definition being implied.

\subsubsection{Adjacency List Representation}

The adjacency list representation is an \texttt{unordered\_map} from VertexIds
to vectors of \texttt{AdjElement}s, or \texttt{AdjList}s.  An adjacency element
is a pointer to an edge and a vertex, which, for simplicity we will represent as
a pair $(v,e)$, but is, in the code, represented as follows

\begin{algorithm}[H]
  \MyStruct{AdjElement}{
    Vertex * vtx\;
    Edge * edge\;
  }

  typedef vector<AdjElement> AdjList\;

  \MyStruct{Graph}{
    unordered\_map<VertexId, AdjList> edges\;
    vector<Vertex> vertices\;
    FeatureIndex index\;
  }
\end{algorithm}

FeatureIndex is really a wrapper class for a 3-d array of vectors defined
\[\texttt{vector<Edge> *** idx}\]
where the ``keys'' to vector of edges are the vertex label, the edge label, and
then the other vertex label. The class also contains $\texttt{add}(Edge)$ and
$\texttt{fetch(Label ul, Label el, Label vl)}$ methods which do exactly as you would
expect.

However this implies an unnecessary directedness for undirected graphs that
results in some problems. So to disambiguate the invariant should be enforced
that the smaller vertex label comes first. So, in $\texttt{fetch(Label ul, Label
  el, Label vl)}$, if $\texttt{ul > vl}$, the method should recursively call
$\texttt{fetch(vl, el, ul)}$.


The AdjacencyList representation also implies a directedness for undirected
graphs that is actually helpful. Since, in the algorithm, we are matching a
QIElem $T$, which usually has an anchored (already-mapped) vertex ($T.p$) and a
free (unmapped) vertex ($T.v$), we can pass the mapped VertexId into the
adjacency list representation get back appropriate candidate edges. Of course
there are several other boolean checks that are necessary. This entire process
is described further by the \textsc{LoopEdges} subroutine
(Algorithm~\ref{alg:loop_edges}).


\textbf{NOTE:} Even though some of the members of the above structs
are declared here as pointers, for simplicity we will abandon the
arrow notation in favor of dot notation, for a greater degree of clarity.


\subsubsection{A Problem}
The biggest problem with this representation is that it has a $O(|V|)$ lookup
time to get a vertex, and so in the current implementation of the
$\texttt{make\_graph}$ function, the vertices are copied over into a hashmap for
faster graph generation, but then the pointers in Edge point to the temporary
hashmap. Not the vertices copied over to the Graph. The recommendation is to use
an $\texttt{unordered\_map}$ for the representation of vertices in the Graph
(i.e. a hash index). That will create a more maintainable data structure. Once
this change is made be sure to propogate the changes through the functions in
\texttt{experiments.cpp}


\subsection{QIElem}

The algorithm uses a structure called a QISequence to represent the query. It
can be thought of as an optimally-ordered vector of the query edges. Spanning
edges are added using a version of Prim's algorithm that resolves conflicts for
edges with equal weight, followed by every newly-induced edge. The QISequence
constructor is defined in~\ref{alg:qi_seq}. It also maintains a representative
temporal approximation for the interval (intersection for \concur, union
otherwise).

The QIElem is a wrapper class for the edge that imposes a direction and
precomputes some useful information. The direction is represented by the member
boolean $\texttt{forwards}$. If the edge is a spanning edge, then the edge is
$\texttt{forwards}$ if and only if the predecessor spanning edge is adjacent to
$u$. If the edge is an inner edge, the edge is forwards if $v$ is an endpoint of
a lower-index edge. An example of this is shown in
Figure~\ref{fig:qi_elem}. This is to show the invariant that spanning edges
point ahead in the QISequence, and inner edges point back to endpoints of
previous edges.

The class provides several accessor methods that get information pertaining to
the head vertex (those called \texttt{vtx\_}*)  and several that get information
pertaining to the tail/parent vertex (those called \texttt{par\_}*).

\begin{figure} \centering
  \begin{tikzpicture}
    \SetGraphUnit{4}
    \Vertex{u}
    \EA(u){v}
    \Edge[style={->,bend left},label=\texttt{fwd}](u)(v)
    \Edge(u)(v)
  \end{tikzpicture}\\[1cm]
  
  \begin{tikzpicture}
    \SetGraphUnit{4}
    \Vertex{v}
    \EA(v){u}
    \Edge[style={->,bend left},label=\texttt{bwd}](u)(v)
    \Edge(u)(v)
  \end{tikzpicture}
  \label{fig:qi_elem}
  \caption{Depiction of imposed ordering on edges. In our representation the
    \texttt{fwd} label means \texttt{forward=true} and the \texttt{bwd} label
    means \texttt{forward=false}}
\end{figure}


\subsection{QISequence}

A QISequence is, at its core, a list of edges (really QIElems) that optimizes
the search order.

\subsubsection{Construction}
For convenience, we also store the vertex degrees and the explicit semantics in
each element of the QISequence (called QIElem) as well as the direction the edge
is traversed in the spanning tree. Note that for inner edges, their
``direction'' is considered to be from higher indices to lower indices.  This is
not a concern for directed graphs. Ultimately, an object-oriented
implementation, storage of these few fields allows for fast implementation of
many of the boolean checks described in \ref{alg:query_proc}.

\begin{algorithm}
  \label{alg:qi_seq}
  \caption{\textsc{MakeQISeq}$(Q,x,m)$}
  \SetAlgoLined
  \KwIn{A query graph $Q = (E,V)$, Explicit semantics $\expVar$, and Implicit semantics
    $\impVar$ }

  \KwOut{The QISequence $\seq$ constructed from $Q$}

  $\seq := \emptyset$\;
  $V_t := \emptyset$\;
  $E_t := \emptyset$\;
  size $:= 0$\;

  Initilize $used$ to a zeroed map from edges to bools\;

  $P := \textsc{Potentials}(E)$ \;
  $e := \textsc{SelectFirstEdge}(P, Q, used)$\;

  $E_t \cup \{e\}, \seq.\texttt{add}(e), used[e] := true$\;
  $V_t \cup \{e.u,e.v\}$\;
  $size++$\;

  \While{$size < |E|$}{
    $P := {\textsc{LoopPotentials}}(Q, V_t, used)$\;
    \If{$|P| > 0$}{
      $e := \textsc{SelectSpanningEdge}(P, used, Q, V_t)$\;

      $E_t \cup \{e\}, \seq.\texttt{add}(e), used[e] := true$\;
      $V_t \cup \{e.u,e.v\}$\;
      $size++$\;
    }

    \ForEach{edge $f$ in $Q[V_t]$}{
      \lIf{$used[f]$}{ \textbf{continue} }
      $\seq.\texttt{add}(f), \texttt{used}[f] = true$\;
    }

    \lIf{$|P| <= 0$}{ \textbf{break} }
  }
  \Return \seq\;

\end{algorithm}

This algorithm uses several important subroutines that determine valid edges
(\textsc{Potentials}, \textsc{LoopPotentials}) and choose a spanning edge
(\textsc{FirstSpanningEdge}, \textsc{SelectSpanningEdge}). \textsc{Potentials}
simply finds the edges with the smallest weight using a Least Of All Time
method.  \textsc{LoopPotentials} finds an unused edge with exactly one endpoint
in \seq.  \textsc{SelectFirstEdge} arbitrarily selects one of the edges with
maximum degree, where \textsc{SelectSpanningEdge} filters by smallest weight,
largest number of edges in $Q[V_t]$ largest degree.  The (unimplemented)
temporal tie-breaker would filter by the smallest intersection between the
predecessor edges and current edge \todo{implement temporal sorting condition in
  QISequence constructor. Perhaps experimentation to determine the optimal
  filtering order.}

\subsubsection{Adding an edge to the QISequence}
Adding to the QISequence is a bit involved. This insertion operation is
$O(|\seq|)$ as we must determine the status of each edge.  We iterate backwards
through the QISequence maintaining zero-initialized booleans \texttt{fw},
\texttt{isR}, and \texttt{anchored}, which become $true$ if the edge is forward,
if its an inner edge, and if an endpoint has been found, respectively. We check
if either endpoint has been found, and once it has we assign the variables
accordingly. This process is detailed in Algorithm~\ref{alg:add}

\begin{algorithm}
  \label{alg:add}
  \caption{$\seq.\texttt{add}(e, Q)$}
  \KwIn{An edge $e$ and the query graph from which the sequence is being
    constructed $Q$}

  \texttt{fw} := false;
  \texttt{isR}:= false;
  \texttt{anchored} := false;
  
  \ForEach{$T \in \seq$}{
    \lIf{isR}{ \textbf{break} }

    \If{ $e.v$ is one of the endpoints of $T.edge$}{
      \texttt{fw}  := \texttt{anchored}\;
      \texttt{isR} := \texttt{anchored}\;
      \texttt{anchored} := true;
    }

    \If{ $e.u$ is one of the endpoints of $T.edge$}{
      \texttt{fw}  := not \texttt{anchored}\;
      \texttt{isR} := \texttt{anchored}\;
      \texttt{anchored} := true;
    }
    
  }
\end{algorithm}

The correctness of the algorithm relies on the fact that the booleans behave as
described above.



\subsubsection{Edge-Weights and Data-Graph Profiling}
In the original quickSI algorithm the profile function estimates the
frequency-per-graph of the edge feature (meaning its triple of vertex label,
edge label, and vertex label.  For a single graph it should simply calculate the
number of feasible candidates for each edge and assign it as edge weight
thereof.  This is used above in QISequence creation
(Algorithm~\ref{alg:qi_seq}), to find an optimal search order that limits the
number of results as soon as possible. Currently this is done by linear search
of the edges resulting in $O(|E(D)||E(Q)|)$ efficiency.  It could/should be done
in $O(|E(Q)|)$ using the \texttt{FeatureIndex} of the data graph $D$. This is
the \texttt{profile} function in the \texttt{quicktsi.cpp} file.


\subsection{Isomorphism}
The Isomorphism is tracked using three maps, \texttt{eiso}, the edge isomorphism
from query EdgeIds to data EdgeIds, \texttt{viso}, the vertex isomorphsim from
query VertexIds to data VertexIds, and \texttt{de\_used}, a map from data
EdgeIds to booleans, where true represents that the edge has been used.

This is where state is stored, and new isomorphic mappings are popped on and off
in standard isomorphism algorithms similar to a stack, but without the easy
interface. In the normal course of the algorithm we use the accessor methods
\texttt{add(QIElem T, AdjElement adj)} and \texttt{remove(QIElem T, AdjElement
  adj)}. These do exactly as you would expect, updating the isomorphic mapping
with pairs $(T.\texttt{vtx\_idx()}, adj.vtx.ident)$, and $(T.edge.ident,
adj.edge.ident)$. This assumes that the pair between $T.\texttt{par\_idx()}$ and
the identifier of the vertex of $adj.edge$ that is not attached to $adj.vtx$ has
already been found (or will be removed in the future. However, for the first
addition(last removal) this is not the case. So we the similar methods
\texttt{first\_add} and \texttt{first\_remove} that take care of this
descrepancy.


\subsection{QuickTSI}

There are two main routines that drive the query, one is
\textsc{QueryProc}~\cite{alg:query_proc}, which creates the QISequence (an
optimized encoding of the query pattern), initializes the isomorphism object
\todo{This is abstracted to three vectors in the QuickSI paper -- an approach
  that I'd prefer for a paper, but this is clearer given the code}
(which keeps track of the node and vertex mappings, as well as which data edges
have been used), and calls other main routine, \textsc{QuickTSI}, which finds
all subgraphs isomorphic to the graph encoded by the QISequence.

\begin{algorithm}
  \label{alg:query_proc}
  \caption{\textsc{QueryProc}$(Q,G)$}
  \SetAlgoLined
  \KwIn{A query graph $Q$, A data graph $D$, Explicit Semantics $x$, and
    Implicit semantics $m$}
  \KwOut{All subgraph ismorphisms of $Q$ in $G$}

  Let $\seq$ be the QISequence for $Q$\;
  
  Initialize the Mapping \iso to $\emptyset$\;

  \textsc{SubgraphSearch}$(Q,D,\iso,0)$\;

  \setcounter{AlgoLine}{0}
  
  \SetKwProg{subroutine}{Subroutine}{}{}
  \subroutine{\textsc{SubgraphSearch}$(\seq,D,\iso,d)$}{
    \eIf{$d = |\seq|$}{
      \textbf{report} $M$\;
    }{
      $T := \seq[d]$\;
      
      \eIf{$d = 0$}{
        $C := \textsc{InitialEdges}(D, T, \iso)$\;
      }{
        $C := \textsc{LoopEdges}(D, T, \iso)$\;
      }
      \ForEach{$e,v \in C$ }{
        \iso.add(t,adj)\;
        \textsc{QuickTSI}$(\seq, D, \iso, d+1)$\;
        \iso.add(t,adj)\;
      }
    }
  }
\end{algorithm}

By nature of the construction of the QISequence, key invariants are that
\textsc{InitialEdges} and \textsc{LoopEdges} return the set of edges that are
valid for the isomorphism for the QISequence traversed so far. Lets look at each
of them separately.

\textsc{InitialEdges} amounts to a rather simple boolean check against the edges
of the Data Graph, $D$, that the labels match, and that the explicit semantics
check out. For additional future pruning power, we include Ullman's
degree-refinement technique. The three conditions can can be expressed by the
simple boolean expression

\[\{e | e \in E(D) \wedge T.l = e.l, \wedge \expLocal(\expVar)(T.edge, e) \wedge
deg(T.vtx) \leq deg(e.v) \}\]

In the implementation, the boolean check is built into the QIElem object,
meaning that for a data edge $e \in E(D)$ and free endpoint $v$ we call
$T.\texttt{match}(e, v, D)$

One observation from looking at the data is that there seems to be a lot of
overhead in the algorithm. This method is one of those places. Currently to find
the first edge, we are iterating through every edge in the data graph performing
some boolean checks on them. One way to speed this up significantly would be to
use the FeatureIndex to get only those edges that exactly match the labels of
the zeroth element of $\seq$. Even better, here we could encorporate the
TemporalIndex to find edges that match the labels and match the explicit
temporal condition.

\begin{algorithm}
  \label{alg:loop_edges}
  \caption{$\textsc{LoopEdges}(D, T, \iso )$}
  \KwIn{A data graph $D$, the QIElem $T$, and the current state $\iso$ }
  \KwOut{A vector of Adjacency Elements, or pairs $(e,v) \in E(D) \times V(D)$}
  \SetAlgoLined

  $es$ := $\emptyset$\;

  \ForEach{adjacency list $alist \in D\texttt{.edges}$}{
    \ForEach{Adjacency Element $(e,v) \in alist$}{
      \lIf{\iso.\texttt{de\_used[}$e$\texttt{]}}{ \textbf{continue}}
      \lIf{not T\texttt{.match}$(e, v, D)$}{\textbf{continue}}

      e\texttt{.push}((e,v))\;
    }
   }

  \Return $es$\;
\end{algorithm}

Unsurprisingly, \textsc{LoopEdges} ensures a similar set of invariants looping
instead through edges coincident with already-mapped edges, that the labels,
explicit semantics, and degrees are copacetic, that the edge hasnt been used in
\iso, that the interval of the edge intersects with the outer interval of the
edge isomorphism, and if the edge is an \textit{inner edge} where both its
endpoints are already mapped, then the endpoints of the data edge are
appropriately assigned.  Let's breakdown each of these pieces.

Taking care of the labels, explicit semantics, and degrees are exactly the same
as in \textsc{InitialEdges} using the $\texttt{match}$ member function of the
QIElem under consideration $T$. However, at this point there is some partial
mapping, so we want to check if the data edge we're considering as the image of
$T$ is in this partial mapping. The \iso object maintains this in the hashmap
\texttt{de\_used} (\textbf{d}ata \textbf{e}dge \textbf{used}). Then we check if
there is a nonempty intersection between the interval maintained by \iso and the
edge to be added. For the \concur semantics, this alone is enough to ensure the
implicit semantics. For the \strongConsec and \weakConsec semantics it is just a
search space pruning method\footnote{For \weakConsec on directed graphs, omit
  the check, since we cannot prune using this information}. So we don't have to
perform the $O(deg(T))$ implicit semantics check on \textit{every} possible
edge. Then we check \iso for whether the edge is an inner edge, and if it is, we
make sure that its second endpoint is mapped appropriately.  And finally we
check the implicit semantics using $\impBool(\impVar)$.

According to the 2012 survey~\cite{2012-VLDB-IsoSurvey}, the reason that this
algorithm is so performant is due to the iteration procedure of
\textsc{LoopEdges}. Unlike other, existing algorithms, at any given search
depth, it only iterates through edges incident to the image of the partial
mapping, as opposed to all edges.  This is done simply via the adjacency-list
representation of the data graph, as demonstrated in
Algorithm~\ref{alg:loop_edges}

\begin{algorithm}
  \label{alg:loop_edges}
  \caption{$\textsc{LoopEdges}(\seq, D, T, \iso)$}
  \KwIn{The QISequence \seq, the data graph $D$, the QIElem $T$, and the partial
    mapping \iso}
  \KwOut{A vector of $(e,v)$ whose addition to \iso is a valid isomorphism for
    $\seq[\seq.\texttt{idxof}(T)] \cup {T} $, i.e. Candidate edge-vertex pairs }

  $E := \emptyset$\;
  
  dat\_vtx\_id := \iso.viso[$T$.p]\;
  
  alist $:= D.\texttt{edges}[dat\_vtx\_id]$\;

  \ForEach{ $(e,v) \in$ alist }{
    
   \If{\iso.\texttt{viso}$[T.v] \geq 0$} {
      \lIf {v != \iso.\texttt{viso}[$T$.v]}{ \textbf{continue} }
    }
    \lIf{\iso.\texttt{de\_used}[e]}{ \textbf{continue} }
    \lIf{$T(\iso) \cap (T(e)) = \emptyset$} { \textbf{continue} }
    \lIf{not $T.\texttt{match}(e,v,D)$} { \textbf{continue} }
    \lIf{not $\impBool(T.\impVar)(E(\iso) \cup e)$} {\textbf{continue}}
        
    es.\texttt{push}$((e,v))$;
  }
  \Return es \;
\end{algorithm}
