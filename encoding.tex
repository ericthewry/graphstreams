\section{Neighborhood Encoding}

This section focuses on encoding methods that we can use to prune the search
space. In a similar vein to the way that GADDI, QuickSI, and GraphQL have
developed subgraph signatures that allow for the search space to be pruned, we
will extend, specifically, the notion of neighborhood subgraph profiles
(GraphQL) to include temporal information. First, lets present the notion of
Neighborhood Subgraphs.

\begin{defn}
  (Neigborhood Subgraph) Given a static graph $G_s$, vertex $v$ and radius $r$,
  the $r$-neighborhood subgraph of vertex $v$, denoted $N(G,v,r)$, consists of
  all vertices within distance $r$ from $v$ and all edges between the
  vertices. Note that $N(G,v,0) = ({v}, \emptyset)$. (Taken from GraphQL)
\end{defn}

This is a fairly basic notion that is easily extensible to consider temporal
information. The first requirement that will help us to prune the search space,
is to force these subgraphs to be contemporary under some condition $c$
(e.g. \textsc{consec} or \textsc{concur}). So it is possible that a given
neighborhood subgraph is contemporary under the given $c$, but that there are
certain subgraphs of the static neighborhood subgraph that are.  So our
extension will return a set of subgraphs, not just a single subgraph.

\begin{defn}
  (Temporal Neighborhood Subgraphs) Given a temporal graph $G$, a vertex $v$, a
  radius $r$, and a condition c, the Temporal Neighborhood Subgraph Set
  $N_c(G,v,r)$ of a vertex $v$ consists of those static neighborhood subgraphs
  that are concurrent under $c$. Note that $\displaystyle\bigcup N_c(G,v,r) =
  N(G,v,r)$.
\end{defn}

However, calculating these temporal $r$-neighborhood subgraphs are expensive, so we
will condense these into a much more lightweight \textit{profile}. For static
$r$-neighborhood subgraphs, we simply lexographically order the labels on the
vertices within the set. Then given a similar ordering of the vertices for the
query graph $Q$, we can prune the search space when the generated $r$-neighborhood
subgraph of the query graph is not a subsequence of the data graph profile.

For temporal $r$-neighborhood subgraphs, we have more information that we can
use to prune the search space. The \textit{temporal profile} of an
$r$-neighborhood subgraph will be a tuple $p = (p_s,p_t)$ of the static profile and
some temporal information. This temporal information will be an interval
constructed from the intervals on the edges of graph. We will henceforth refer
to $p_S$ as the semantic profile, and $p_t$ as the profile interval.

How exactly this profile interval will be constructed is closely related to the
contemporaneity condition $c$, and will be user defined.  For now we will notate
this as $c.p$, to signify that it is the folding protocol defined for the
contemporaneity condition $c$. For example in the \textsc{consec} case we will
take the union of the intevals on the edges, in the \textsc{t-resp} case we will
take interval equal to $({t_s}_{min}, \infty)$, and in the \textsc{concur} case
we will take the intersection of the intervals on the edges.

Once we have this information, we will compare the temporal $r$-neighborhood
profiles of the query graph and the data graph. The semantic condition will be
that the query semantic profile must be a subsequence of the data semantic
profile, where the interval condition will just be that the intersection must be
nonempty.  If there is no query interval specified, then we will consider the
interval to be $(\infty, \infty)$. This is defined formally below

\begin{defn}
  The temporal profile $p$ of a graph $G = (V,E)$ given some contemporaneity condition
  $c$ is defined to be a tuple $(p_s, p_t)$ where
  \begin{itemize}
  \item $p_S$ \emph{\texttt{:=} \, \texttt{toSortedList}} \; $\{L(v) | v \in V \}$,
    and
  \item $p_t$ \emph{\texttt{:=} \, \texttt{foldr1}} \; $c.p$ \; $\{T(e) | e \in E\}$, where
    \texttt{foldr1} accumulates $c.p$ accross the  given set, assuming the set
    is non-empty. If the set is empty then let $p_t$ be $(\infty,\infty)$.
  \end{itemize}
\end{defn}

