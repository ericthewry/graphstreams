\section{Edge Isomorphism via Co-incidence Interval Graph}

In this section we want to enumerate the basics of the interval graph method by
expanding on the matches $h$ we defined above in Definition~\ref{defn:match}. 

We introduce the concept of the {\bf {\em co-incidence interval graph}, a
  derived static graph representing the temporal and incidence relationships
  between adjacent edges of a temporal graph. This interval graph is a lossless
  encoding (shown to be bijective) of the full temporal graph $G$, which allows
  us to run existing algorithms to find appropriate patterns.}

\begin{defn}
  \label{def:ci_graph}
  The co-incidence interval graph, $\I_G^c$ of a graph $G = (V_G,E_G, L_G)$ under
  implicit temporal constraint $c \in \impTC$, is a graph $\I_G^c = (E, \E)$, where its 
  node set is $E$,  the edge set of  graph $G$,  and it is edge set $\E \subseteq E^2$ is
  the set of ``meta-edges'' between elements in $E$. For any pair $e_1, e_e \in E$, 
  $(e_1,e_2) \in \E$ if  
  \begin{itemize}
    \item $e_1$ and $e_2$ share an endpoint, e.g., $\{\pi_1(e_1), \pi_2(e_1) 
    	\cap \pi_1(e_2), \pi_2(e_2) \not= \emptyset$,  and
    \item $\impBool(c,\{e_1,e_2\}) = True$
  \end{itemize}
\end{defn}

The generic nature of this definition of the co-incidence interval graph accommodate any 
implicit temporal semantics defined earlier.

\todo[inline]{example}


 Let $I_c$ be the function that computes this co-incidence interval graph, given an implicit temporal constraint $c$, e.g., $I_c(G) = \I_G^c$.

\begin{lemma}
  \label{lem:ci_biject}
  Given an implicit temporal constraint $c$, the function $I_c$ is a bijection
  over unlabelled\footnote{This is a trivial extension that only serves to
    complicate the proof. The lemma indeed holds for labelled graphs as well.}
  graphs $G$ with $\delta(G) \geq 1$.
\end{lemma}

\begin{proof}[Proof] Surjectivity follows the definition of function $I_c$, so,
  we will only show Injectivity.
  
  \textit{Injectivity}. All of the information necessary to restore the temporal
  graph $G$ is stored in the edge-set $E$, since there is no vertex that is not
  an endpoint of an edge. Thus, if $I_c(G) = I_c(H)$, then $V(I_c(G)) = E(G) =
  E(H) = V(I_c(H))$, hence $G = H$.
\end{proof}

\noindent \textbf{Remarks:}
\begin{itemize}
  \item Note that we didn't use the edge set of $\I^c_G$ at all in the proof
    above. This is because all of the structural information needed to describe
    the graph $G$ is stored in the edge set (labels are handled by an external
    map $L$).
  \item Note that Lemma~\ref{lem:ci_biject} only holds for graphs with
    $\delta(G) \geq 1$. This is because if there is a vertex that has degree
    zero, there is no edge that knows about it. This could be solved if you
    wanted to keep track of these vertices in $\I^c_G$. (Its also very unlikely
    for interesting large graphs for singletons to be of any use or
    importance. They will only be returned in trivial queries such as the empty
    graph or singletons).
\end{itemize}

\begin{corollary}
  Given temporal graphs $G = (V_G, E_G)$ and $H = (V_H, E_H)$ such that
  $\delta(G) \geq 1$ and $\delta(H) \geq 1)$, an implicit constraint $c$, and
  an isomorphism $f_c : V \to V$, then $I_c^{-1} \circ f_c \circ I_c : E \to E$
  is a graph isomorphism if $f_c$ uniquely maps labels.
\end{corollary}

\begin{proof}
  Since $I_c$ losslessly encodes edges as vertices, $f_c$ is really bijection
  between edge sets, that preserves the temporal coincidence relationship
  between edges.  For arbitrary $e,e' \in E_G$ and $f,f'\in E_H$ such that
  $f_c(e) = f$ and $f_c(e') = f'$ show that if and only if $e$ and $e'$ are
  coincident in $G$, $f$ and $f'$ are coincident in $H$.

  For vertex $e \in V(\I_G^c)$, we consider it to be a vertex with labels
  corresponding to its declaration in $G$. Hence the label of $e$ is in $(V^2
  \times T^2)$. So we let $L_{\I_G^c} = \left(u, v \right)$. Then, since $f_c$
  uniquely maps labels, $f_c(\I_G^C)$ will have preserved all adjacencies that
  violate the condition $c$ (as well as those that do not).
\end{proof}

Algorithm~\ref{alg:ci_graph} outlines the construction of a coincidence interval
graph. Its complexity is $O(|E|d_{\max}(G))$ in the edge-relational
representation of the graph).

Algorithm~\ref{alg:ci_graph_inv} depicts the procedure for restoring the
original temporal graph given a co-incidence interval graph.

\begin{algorithm}
  \label{alg:ci_graph}
  \caption{\textsc{MakeCoincidenceInterval($G$, $c$)}, equivalently $I_C(G)$}
  \SetAlgoLined
  \KwIn{A temporal graph $G = (V_G,E_G, L_G)$, a implicit temporal constraint $c$}
  \KwOut{The coincidence interval graph $\I^c_G$}

  $\E = \emptyset$\;
  \ForEach{edge pair $e , f   \in E_G$}{
    \If{$\tau_B(c)(\{T(e),T(f)\})$}{
      Add meta-edge $(e,f)$ to $\E$\;
    }
  }
  \Return $(E_G, \E)$\;
\end{algorithm}

\begin{algorithm}
  \label{alg:ci_graph_inv}
  \caption{\textsc{UnmakeCoincidenceInterval($I_G^c$, $c$)}, equivalently $I_c^{-1}(\I_G^c)$ }
  \KwIn{The co-incidince interval graph $\I_G^c = (E, \E)$ }
  \KwOut{Temproal graph $G$}
  $E_G = E$\;
   $V_G = \emptyset$\;
  
  \ForEach{$e \in E_G$}{
    $V = V \cup \{\pi_1(e), \pi_2(e)\}$\;
  }
  \Return $(V_G,E_G)$\;
\end{algorithm}


