\section{Edge Isomorphism}

In this section we want to enumerate the basics of the interval graph
method by expanding on the matches $h$ we defined above in
Definition~\ref{defn:match}. Here we introduce the concept of the
Coincidence Interval Graph, $\I_G^c$ of a temporal graph $G$. This
interval graph is a lossless encoding (shown to be bijective) of the
full temporal graph $G$, which allows us to run existing algorithms to
find appropriate patterns. In order to do this, we first need to
introduce the concept of a temporal condition $c$.

\begin{defn}
  A temporal condition $c$ is a combination of a logical forumla, $c.log$ that
  describes what it means for two edges to be ``contemporary,'' a function $c.f
  : E^2 \to \textbf{2}$ that determines whether a pair of edges satisfy $c.log$
  , and a join policy $c.p: T^2 \to T$, which determines how to combine two
  intervals for search space pruning.
\end{defn}

\begin{defn}
  The Coincidence Interval Graph, $\I_G^c$ of a graph $G$ under
  temporal condition $c$, is a tuple $\I_G^c = (E(G), \E)$, where
  $E(G)$ is the edge set of the graph $G$ (and the vertex set of
  $I_G^c$, and $\E \subseteq E(G)^2$ is the set of ``meta-edges'' between the edges of
  $G$ (vertices of $I_G^c$). There is an edge between $e,f \in E(G)$
  exactly when
  \begin{itemize}
    \item $e$ and $f$ share an endpoint, and
    \item $c.f(e,f)$ returns \emph{True}
  \end{itemize}

  Let the function that computes this graph under the condition $c$ be $I_c$.
\end{defn}

The generic nature of this definition of the Coincidence Interval Graph allows
for the user to define textit{how} the contemporaneity of the query can be
defined. Simply, this can be extended to define the \textit{\bf implicit
  temporal semantics} of the resultant query. In practice we will use the
functions \textsc{consec} and \textsc{concur} (enforcing definitions
\ref{def:consec} and \ref{def:concur} respectively) most frequently. But one
could also imagine infinitely nuanced definitions. Another that is adapted from
one commonly used for time-respecting paths (Kempe et al) is \textsc{t-resp},
wherein the graph must be weakly temporally connected (citation). \todo{example}

\begin{lemma}
  \label{lem:ci_biject}
  The function $I_c$ that constructs the Coincidence Interval Graph given some
  contemporaneity condition $c$ is a bijection for graphs with no singleton
  components.
\end{lemma}

\begin{proof}
  \textit{injective}. Want to show that for any $c$, $I_c(G) = I_c(H)$ implies
  $G = H$ for any two graphs $G$ and $H$.  \todo[inline]{prove injectivity}
  \textit{surjective}. Want to show that for every interval graph $\I^c_G$,
  there exists a graph $G$, such that $I_c(G) = \I^c_G$.  \todo[inline]{prove
    surjectivity}
\end{proof}

The construction of this graph (as defined in Algorithm
\ref{alg:ci_graph}) is a fairly straightforward algorithm
(and in fact is $O(|E|d_{\max}(G))$ in the edge-relational representation of the
graph). A quick corollary of Lemma \ref{lem:ci_biject} is that given a condition
$c$, if there exists some isomorphism $f_c : \I^c_G \to \I^c_H$, then $I_c^{-1}
\circ f_c \circ I_c : G \to H$ is also an isomorphism.

\begin{algorithm}
  \label{alg:ci_graph}
  \caption{\textsc{MakeCoincidenceInterval($G$, $c$)}, equivalently $I_C(G)$}
  \SetAlgoLined
  \KwIn{A temporal graph $G = (V,E)$, a contemporaneity condition $c$}
  \KwOut{The interval coincidence graph $\I^c_G$}

  Initialize $\E$ to $\emptyset$\;
  \ForEach{edge pair $e = (w,u), f =  (u,v) \in E$}{
    Add meta-edge $(e,f)$ to $\E$\;
  }
  \Return $(E, \E)$\;
\end{algorithm}

\begin{algorithm}
  \label{alg:ci_graph_inv}
  \caption{\textsc{UnmakeCoincidenceInterval($I_G^c$, $c$)}, equivalently $I_c^{-1}(\I_G^c)$ }
  \KwIn{The coincidince interval graph $\I_G^c$, $c$}
  \KwOut{The original graph $G$}

  Initialize $V$ and $E$ to $\emptyset$\;
  
  \ForEach{edge $(u,v) \in V(I_G^c)$}{
    Add $u$ and $v$ to $V$\;
    Add $(u,v)$ to $E$\;
  }
  \Return $(V,E)$\;
\end{algorithm}
