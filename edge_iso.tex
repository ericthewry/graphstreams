\section{Edge Isomorphism}

In this section we want to enumerate the basics of the interval graph method by
expanding on the matches $h$ we defined above in
Definition~\ref{defn:match}. Here we introduce the concept of the co-incidence
interval graph, a derived static graph representing the temporal and incidence
relationships between th e edges of a temporal graph. This interval graph is a
lossless encoding (shown to be bijective) of the full temporal graph $G$, which
allows us to run existing algorithms to find appropriate patterns. However, what
kind of temporal information it captures is determined by an \textit{temporal
  condition} $c$, which determines which version of implicit temporal semantics
is being used.

\begin{defn}
  A temporal condition $c$ is an element of the condition set $\mathcal{C} =
  \{\textsc{concur}, \textsc{Sconsec}, \textsc{Wconsec}\}$,that determine the
  temporal logic to be applied. This is essentially an \emph{\texttt{enum}} type that
  will allow the user to switch between the different implicit semantics as
  defined in Definitions \ref{def:concur} and \ref{def:consec}.
\end{defn}

\begin{defn}
  The function $\tau_B : \mathcal{C} \times P(T^2) \to \textbf{2}$ determines
  whether a set of of edges adhere to the given temporal condition.
\end{defn}

\begin{defn}
  \label{def:ci_graph}
  The co-incidence interval graph, $\I_G^c$ of a graph $G = (V,E)$ under
  temporal condition $c$, is a tuple $\I_G^c = (E, \E)$, where $E$ is the edge
  set of the graph $G$ (and the vertex set of $I_G^c$, and $\E \subseteq E^2$ is
  the set of ``meta-edges'' between the edges of $G$ (vertices of
  $I_G^c$). There is an edge between $e,f \in E$ exactly when
  \begin{itemize}
    \item $e$ and $e'$ share an endpoint, and
    \item $\tau_B(c,\{e,e'\})$ returns \emph{True}
  \end{itemize}

  Let the function that computes this graph under the condition $c$ be $I_c$.
\end{defn}

The generic nature of this definition of the coincidence interval graph allows
for the user to define \textit{how} the contemporaneity of the query can be
defined. Simply, this can be extended to define the \textit{\bf implicit
  temporal semantics} of the resultant query.

\todo[inline]{example}

\begin{lemma}
  \label{lem:ci_biject}
  Given a contemporaneity condition $c$, the function $I_c$ is a bijection over
  graphs with $\delta(G) \geq 1$.
\end{lemma}

\begin{proof}[Proof] Since surjectivity is evident from
  Definition~\ref{def:ci_graph}i we will only show Injectivity.
  
  \textit{Injectivity}. All of the information necessary to restore a graph is
  stored in the edge-set, since there is no vertex that is not an endpoint of an
  edge. Thus, if $I_c(G) = I_c(H)$, then $V(I_c(G)) = E(G) = E(H) = V(I_c(H))$,
  and so $G = H$.
\end{proof}

\clearpage
\noindent \textbf{Remarks:}
\begin{itemize}
  \item Note that we didn't use the edge set of $\I^c_G$ at all in the proof
    above. This is because all of the structural information needed to describe
    the graph $G$ is stored in the edge set (labels are handled by an external
    map $L$).
  \item Note that Lemma~\ref{lem:ci_biject} only holds for graphs with
    $\delta(G) \geq 1$. This is because if there is a vertex that has degree
    zero, there is no edge that knows about it. This could be solved if you
    wanted to keep track of these vertices in $\I^c_G$. (Its also very unlikely
    for interesting large graphs for singletons to be of any use or
    importance. They will only be returned in trivial queries such as the empty
    graph or singletons).
\end{itemize}

The construction of this graph (as defined in Algorithm \ref{alg:ci_graph}) is a
fairly straightforward algorithm (and in fact is $O(|E|d_{\max}(G))$ in the
edge-relational representation of the graph). A quick corollary of Lemma
\ref{lem:ci_biject} is that given a condition $c$, if there exists some
isomorphism $f_c : \I^c_G \to \I^c_H$, then $I_c^{-1} \circ f_c \circ I_c : G
\to H$ is also an isomorphism.

\begin{algorithm}
  \label{alg:ci_graph}
  \caption{\textsc{MakeCoincidenceInterval($G$, $c$)}, equivalently $I_C(G)$}
  \SetAlgoLined
  \KwIn{A temporal graph $G = (V,E)$, a contemporaneity condition $c$}
  \KwOut{The interval coincidence graph $\I^c_G$}

  Initialize $\E$ to $\emptyset$\;
  \ForEach{edge pair $e = (w,u), f =  (u,v) \in E$}{
    \If{$\tau_B(c)(\{T(e),T(f)\})$}{
      Add meta-edge $(e,f)$ to $\E$\;
    }
  }
  \Return $(E, \E)$\;
\end{algorithm}

\begin{algorithm}
  \label{alg:ci_graph_inv}
  \caption{\textsc{UnmakeCoincidenceInterval($I_G^c$, $c$)}, equivalently $I_c^{-1}(\I_G^c)$ }
  \KwIn{The coincidince interval graph $\I_G^c$, $c$}
  \KwOut{The original graph $G$}

  Initialize $V$ and $E$ to $\emptyset$\;
  
  \ForEach{edge $(u,v) \in V(I_G^c)$}{
    Add $u$ and $v$ to $V$\;
    Add $(u,v)$ to $E$\;
  }
  \Return $(V,E)$\;
\end{algorithm}


