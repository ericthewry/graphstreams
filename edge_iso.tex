\section{Edge Isomorphism}

In this section we want to enumerate the basics of the interval graph
method by expanding on the matches $h$ we defined above in
Definition~\ref{defn:match}. Here we introduce the concept of the
Coincidence Interval Graph, $\I_G^c$ of a temporal graph $G$. This
interval graph is a lossless encoding (shown to be bijective) of the
full temporal graph $G$, which allows us to run existing algorithms to
find appropriate patterns.

\begin{defn}
  The Coincidence Interval Graph, $\I_G^c$ of a graph $G$ under
  temporal condition $c$, is a tuple $\I_G^C = (E(G), \E)$, where
  $E(G)$ is the edge set of the graph $G$ (and the vertex set of
  $I_G^c$, and $\E \subseteq E(G)^2$ is the set of ``meta-edges'' between the edges of
  $G$ (vertices of $I_G^c$). There is an edge between $e,f \in E(G)$
  exactly when
  \begin{itemize}
    \item $e$ and $f$ share an endpoint, and
    \item $c(e,f)$ returns \emph{True}
  \end{itemize}

  Where $c: E^2 \to \textbf{2}$ is a function that allows the user to
  control the conditions under which two edges are ``contemporary''.
  Let the function that computes this graph under the condition $c$ be $I_c$.
\end{defn}

The generic nature of this definition of the Coincidence Interval
Graph allows for the user to define textit{how} the contemporaneity of
the query can be defined. Simply, this can be extended to define the
\textit{\bf implicit temporal semantics} of the resultant query. In
practice we will use the functions \textsc{consec} and \textsc{concur}
(enforcing definitions \ref{def:consec} and \ref{def:concur}
respectively) most frequently. But one could also imagine infinitely
nuanced definitions. Another that is adapted from one commonly used
for time-respecting paths (Kempe et al) is \textsc{t-resp}, wherein
the graph must be weakly temporally connected (citation).

\begin{lemma}
  \label{lem:ci_biject}
  The function $I_c$ that constructs the Coincidence Interval Graph
  given some contemporaneity condition $c$ is a bijection.
\end{lemma}

\begin{proof}
  \textit{injective}. Want to show that for any $c$, $I_c(G) = I_c(H)$ implies $G = H$
  for any two graphs $G$ and $H$.
  \todo[inline]{prove injectivity}
  \textit{surjective}. Want to show that for every interval graph
  $\I^c_G$, there exists a graph $G$, such that $I_c(G) = \I^c_G$.
  \todo[inline]{prove surjectivity}
\end{proof}

The construction of this graph (as defined in Algorithm
\ref{alg:ci_graph}\todo{write this alg}) is a fairly
straightforward algorithm (and in fact is $O(|E|d_{\max}(G))$ in the
edge-relational representation of the graph). A quick corollary of
Lemma \ref{lem:ci_biject} is that given a condition $c$, if there
exists some isomorphism $f_c : \I^c_G \to \I^c_H$, then $I_c^{-1}
\circ f_c \circ I_c : G \to H$ is also an isomorphism.
