\subsection{Simple Modification of $\textsc{IsJoinable}$}
\label{sec:naive_joinable}

Now we want to similarly consider a more general form of the extension to VF2
\cite{2004-PAMI-VF2} done in \cite{2016-arXiv-TemporalIso}, in which the authors
somewhat informally presented a $Ti\&To$ algorithm in which they considered the
the temporal information as they considered the Topographical information by
extending the semantic function built into VF2. This is the \textsc{IsJoinable}
subroutine introduced in Algorithm~\ref{alg:gen_query_proc} following the
convention established in \cite{2012-VLDB-IsoSurvey}. To mirror this simple
extension, we will enforce the condition $\tau_B$ for every new edge introduced,
and reject the edge if one of them fails. Essentially, if a query vertex $u'$ is
adjacent to $u$ and has already been matched, then it ensures that there is a
corresponding edge in the data graph (with matching label if necessary). In
\cite{2004-PAMI-VF2}, they maintain the dates of previously accessed nodes to
assure that the current node maintains the \textsc{Wconsec} condition. However,
since we are finding a mapping between edges (which contains a mapping between
nodes), our mapping contains all of the edges that have been used, so we already
have the relevant information.

This algorithm is presented in Algorithm~\ref{alg:naive_isjoinable}. First, we
will need some notation. Let $M_Q$ be the domain-so-far, and let $M_G$ be the
image-so-far. I.e. $M_Q \assign \texttt{map } \pi_1 M$ and $M_G \assign
\texttt{map } \pi_2 M$.  The algorithm relies on the invariant that the
mapping-so-far is contemporary with respect to some implicit semantics $\impVar$.

\begin{algorithm}
  \label{alg:naive_isjoinable}
  \caption{\textsc{IsJoinable}$(Q,T_q,\impVar, \expVar,G,e,f,M)$}
  
  \KwIn{A query graph $Q$, $T_q$ a time interval, $\impVar$ an implicit
    semantics, $\expVar$ an explicit semantics, a data graph $G$, $e \in E(Q)$,
    $f \in E(G)$, , and $M \in P(E(Q) \times E(Q))$ the mapping so far}

  \KwOut{A boolean representing whether we can safely add the pair $e \mapsto f$
    to $M$}
  
  \ForEach{$e' \in (Pred(e) \cup Succ(e)) \cap M_Q$ }{

    Let $f' \assign M(e)$\;
    \If{not $\expLocal(\expVar)(T_q,M_Q \cup \{e\}, M_G \cup \{f\})$ }{
      \Return \emph{False}\;
    }
    \If{not $\impBool(\impVar)(M_Q \cup \{e\}, M_G \cup \{f\})$}{
      \Return \emph{False}\;
    }
  }
  
  \Return \emph{True}\;
  
\end{algorithm}


This is obviously going to significantly reduce the search space from the naive
approach presented in section \ref{sec:postcondition}. This will prune very
early the execution large sections of the tree that will not be searchable since
they rely on a non-consecutive or non-intersecting edge.
