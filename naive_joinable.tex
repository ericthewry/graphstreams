\subsection{Simple Modification of $\textsc{IsJoinable}$}
\label{sec:naive_joinable}

Now we want to similarly consider a more general form of the extension to VF2
\cite{2004-PAMI-VF2} done in \cite{2016-arXiv-TemporalIso}, in which the authors
somewhat informally presented a $Ti\&To$ algorithm in which they considered the
the temporal information as they considered the Topographical information by
extending the semantic function built into VF2. This is the \textsc{IsJoinable}
subroutine introduced in Algorithm~\ref{alg:gen_query_proc} following the
convention established in \cite{2012-VLDB-IsoSurvey}. To mirror this simple
extension, we will enforce the condition $\tau_B$ for every new edge introduced,
and reject the edge if one of them fails. Essentially, if a query vertex $u'$ is
adjacent to $u$ and has already been matched, then it ensures that there is a
corresponding edge in the data graph (with matching label if necessary). In
\cite{2004-PAMI-VF2}, they maintain the dates of previously accessed nodes to
assure that the current node maintains the \textsc{Wconsec} condition. However,
since we are finding a mapping between edges (which contains a mapping between
nodes), our mapping contains all of the edges that have been used, so we already
have the relevant information.

This algorithm is presented in Algorithm~\ref{alg:naive_isjoinable}. First, we
will need some notation. Let $M_Q$ be the domain-so-far, and let $M_G$ be the
image-so-far. I.e. $M_Q := \texttt{map fst} M$ and $M_G := \texttt{map snd} M$.
The algorithm relies on the invariant that the mapping-so-far is contemporary
with respect to some condition $c$.

\begin{algorithm}
  \label{alg:naive_isjoinable}
  \caption{\textsc{IsJoinable}$(Q,T_q,G,e,f,M,c)$}
  \KwIn{A query graph $Q$, $T_q$ a time interval, a data graph $G$, $e \in V(Q)$, $f \in G(Q)$, $c$ a
    contemporaneity condition, and $M \in P(E(Q)\times G(Q))$ the mapping so
    far}
  \KwOut{A boolean representing whether we can safely add the pair $e \mapsto f$
    to $M$}
  
  Let $e \texttt{:=} (u,u')$ and $f \texttt{:=} (v,v')$\;
  
  \ForEach{$e' \in (Pred(e) \cup Succ(e)) \cap M_Q$ }{

    \eIf{$\exists f' \in (Pred(e) \cup Succ(e)) \cap M_G. (e \mapsto f') \in M$}{
      \eIf{$c \neq \emptyset$ and $T_q \neq \emptyset$ }{
        \If{not $\tau(c)(\{T_q\} \cup \{T(f'') | f'' \in M_G \cup \{f\}\})$}{
          \Return \emph{False}\;
        }
      }{
        \If {$c \neq \emptyset$ and not $\tau(c)(\{T(f'') | f'' \in M_G \cup \{f\}\})$}{
          \Return \emph {False}\;
        }
        \If{$T_q \neq \emptyset$ and not $\tau(\texttt{consec})(\{T(f),T_q\})$}{  
          \Return \emph{False}\;
        }
      }
    }{
      \Return \emph{False}\;
    }
    
    \Return \emph{True}\;
  }
\end{algorithm}


This is obviously going to significantly reduce the search space from the naive
approach presented in section \ref{sec:postcondition}. This will prune very
early the execution large sections of the tree that will not be searchable since
they rely on a non-consecutive or non-intersecting edge.
