\section{Implementation Structure}
\label{sec:structure}

Since Ullman's original search-space pruning algorithm~\cite{1976-ACMJ-Ullman}
published in 1976 there has been an influx of new algorithms attempting to find
tighter subspaces and improved search orders, as well as storing partial results
in graph indexes to allow for faster access.  The most recent, and fastest
algorithms have been in the last 8 years. Notably, these are
QuickSI~\cite{2008-VLDB-QuickSI}, GraphQL~\cite{2008-SIGMOD-GraphQL},
TurboIso~\cite{2013-SIGMOD-TurboISO}, BoostIso~\cite{2015-VLDB-BoostIso}, and
DualIso~\cite{2014-IEEE-DualIso}.

A 2012 comparison of existing algorithms~\cite{2012-VLDB-IsoSurvey} concluded
that QuickSI, and GraphQL were the fastest from among other algorithms including
GADDI, SPath~\cite{2010-VLDB-SPath}, and VF2~\cite{2004-PAMI-VF2}. It created a
common framework for all of the algorithms, that allowed for a more
comprehensive understanding of the way in which these graphs are being
queried. It is essentially broken up into four steps. \textsc{FilterCandidates},
which performs a label search on a given vertex.  Once this has been performed
for all vertices, the recursive subroutine \textsc{SubgraphSearch} is
called. Within this routine, there is the function \textsc{NextQueryVertex},
which determines the search order of the query graph, \textsc{IsJoinable}, which
determines whether the proposed match is actually viable, \textsc{UpdateState},
which updates the mapping with the joinable pair, then the recursive call, and
finally, \textsc{RestoreState}, which removes the pair from the mapping. This is
explicitly stated in Algorithm~\ref{alg:gen_query_proc}.

\begin{algorithm}
  \label{alg:gen_query_proc}
  \caption{\textsc{GenericQueryProc}$(Q,G)$}
  \SetAlgoLined
  \KwIn{A query graph $Q$, A data graph $G$}
  \KwOut{All subgraph ismorphisms of $Q$ in $G$}

  Initialize the Mapping $M$ to $\emptyset$\;
  \ForEach{$u \in V(Q)$}{
    $\Phi(u)$ \texttt{:=} \textsc{FilterCandidates}$(G,Q,u, \cdots )$\;
    \If{$\Phi(u) = \emptyset$}{ \Return{} \; }
  }

  \textsc{SubgraphSearch}$(Q,G,M,\Phi, \cdots)$\;

  \setcounter{AlgoLine}{0}
  \SetKwProg{subroutine}{Subroutine}{}{}
  \subroutine{\textsc{SubgraphSearch}$(Q,G,M,\Phi, \cdots)$}{
    \eIf{$|M| = |V(Q)|$}{
      \textbf{Report} $M$\;
    }{
      $u$ \texttt{:=} \textsc{NextQueryVertex} $(\cdots)$\;
      $\Phi'(u)$ \texttt{:=} \textsc{RefineCandidates} $(M,u, \Phi(u), \cdots)$\;
      \ForEach{$v \in \Phi'(u)$ that is not yet matched}{
        \If{\textsc{IsJoinable}$(Q,G,u,v, \cdots )$}{
          \textsc{UpdateState}$(M,u,v, \cdots )$\;
          \textsc{SubgraphSearch}$(Q,G,M, \cdots)$\;
          \textsc{RestoreState}$(M,u,v, \cdots )$\;
        }
      }
    }
  }
\end{algorithm}

In the next section we will detail how to go about developing this framework for
existing graphs. Specifically, how we can use temporal information to further
restruct the search space for existing algorithmic paradigms.

\subsection{Temporal Postcondition}
\label{sec:postcondition}

Similar to the way in which \cite{2016-arXiv-TemporalIso} developed several
naive versions of the VF2~\cite{2004-PAMI-VF2} algorithm to include the basics
of the \textsc{Wconsec} temporal semantics. We will consider a similar algorithm
to the \textit{To-Ti}, algorithm where the topographical information is
considered before the temporal information. Here, we simply filter the results
of any implementation of \textsc{GenericQueryProc} with $\tau_b \circ T \circ
E$. We get this naive algorithm in Algorithm~\ref{alg:naive_temp}.

\begin{algorithm}
  \label{alg:naive_temp}
  \caption{\textsc{TopTimeQuery}$(<Q,T_q>, G, c)$}
  \KwIn{A temporal query graph $Q$, a time range $T_q$, a
    data graph $G$, and a temporal condition $c$}
  \KwOut{The set of patterns obeying the temporal semantic $c$ matching $Q$ in $G$ }

  $R$ \texttt{:=} \textsc{GenericQueryProc}$(Q,G)$\;
  
  \ForEach{$g \in R$}{

    \eIf{$c \neq \emptyset$ and $T_q \neq \emptyset$}{
      \If{ not $\tau_b(c)(\{T(e) | e \in E(G)\} \cup \{T_q\})$}{
        $R$\texttt{.remove} g\;
        \textbf{next}\;
      }
    }{

      \If{$c \neq \emptyset$ and not $\tau_b(c)(\{T(e) | e \in E(G)\}$}{
        $R$\texttt{.remove} g\;
        \textbf{next}\;
      }
      
      \If{$T_q \neq \emptyset$ and not $\forall e \in
        E(g).\tau_B(\textsc{concur})\{T(e),T_q\}$}{ $R$\texttt{.remove} g\;
        \textbf{next}\; } } \ForEach{$e' \in V(g)$}{ Let $e \in V(Q)$ be the
      vertex mapped to $e'$\; \If{$T(e) \cap T(e') = \emptyset$}{
        $R$\texttt{.remove} g\; \textbf{break}\; } } } \Return $R$\;
\end{algorithm}

This algorithm simply filters out those result graphs that violate the temporal
semantics and/or the time window $T_q$.  When both $c$ and $T_q$ are given,
the potential result graph must take $T_q$ into consideration when checking if
the potential result graph obeys $c$.


\subsection{Simple Modification of $\textsc{IsJoinable}$}
\label{sec:naive_joinable}

Now we want to similarly consider a more general form of the extension to VF2
\cite{2004-PAMI-VF2} done in \cite{2016-arXiv-TemporalIso}, in which the authors
somewhat informally presented a $Ti\&To$ algorithm in which they considered the
the temporal information as they considered the Topographical information by
extending the semantic function built into VF2. This is the \textsc{IsJoinable}
subroutine introduced in Algorithm~\ref{alg:gen_query_proc} following the
convention established in \cite{2012-VLDB-IsoSurvey}. To mirror this simple
extension, we will enforce the condition $\tau_B$ for every new edge introduced,
and reject the edge if one of them fails. Essentially, if a query vertex $u'$ is
adjacent to $u$ and has already been matched, then it ensures that there is a
corresponding edge in the data graph (with matching label if necessary). In
\cite{2004-PAMI-VF2}, they maintain the dates of previously accessed nodes to
assure that the current node maintains the \textsc{Wconsec} condition. However,
since we are finding a mapping between edges (which contains a mapping between
nodes), our mapping contains all of the edges that have been used, so we already
have the relevant information.

This algorithm is presented in Algorithm~\ref{alg:naive_isjoinable}. First, we
will need some notation. Let $M_Q$ be the domain-so-far, and let $M_G$ be the
image-so-far. I.e. $M_Q := \texttt{map fst} M$ and $M_G := \texttt{map snd} M$.
The algorithm relies on the invariant that the mapping-so-far is contemporary
with respect to some condition $c$.

\begin{algorithm}
  \label{alg:naive_isjoinable}
  \caption{\textsc{IsJoinable}$(Q,T_q,G,e,f,M,c)$}
  \KwIn{A query graph $Q$, $T_q$ a time interval, a data graph $G$, $e \in V(Q)$, $f \in G(Q)$, $c$ a
    contemporaneity condition, and $M \in P(E(Q)\times G(Q))$ the mapping so
    far}
  \KwOut{A boolean representing whether we can safely add the pair $e \mapsto f$
    to $M$}
  
  Let $e \texttt{:=} (u,u')$ and $f \texttt{:=} (v,v')$\;
  
  \ForEach{$e' \in (Pred(e) \cup Succ(e)) \cap M_Q$ }{

    \eIf{$\exists f' \in (Pred(e) \cup Succ(e)) \cap M_G. (e \mapsto f') \in M$}{
      \eIf{$c \neq \emptyset$ and $T_q \neq \emptyset$ }{
        \If{not $\tau(c)(\{T_q\} \cup \{T(f'') | f'' \in M_G \cup \{f\}\})$}{
          \Return \emph{False}\;
        }
      }{
        \If {$c \neq \emptyset$ and not $\tau(c)(\{T(f'') | f'' \in M_G \cup \{f\}\})$}{
          \Return \emph {False}\;
        }
        \If{$T_q \neq \emptyset$ and not $\tau(\texttt{consec})(\{T(f),T_q\})$}{  
          \Return \emph{False}\;
        }
      }
    }{
      \Return \emph{False}\;
    }
    
    \Return \emph{True}\;
  }
\end{algorithm}


This is obviously going to significantly reduce the search space from the naive
approach presented in section \ref{sec:postcondition}. This will prune very
early the execution large sections of the tree that will not be searchable since
they rely on a non-consecutive or non-intersecting edge.

