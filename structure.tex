\section{Implementation Structure}
\label{sec:structure}

Since Ullman's original search-space pruning algorithm~\cite{1976-ACMJ-Ullman}
published in 1976 there has been an influx of new algorithms attempting to find
tighter subspaces and improved search orders, as well as storing partial results
in graph indexes to allow for faster access.  The most recent, and fastest
algorithms have been in the last 8 years. Notably, these are
QuickSI~\cite{2008-VLDB-QuickSI}, GraphQL~\cite{2008-SIGMOD-GraphQL},
TurboIso~\cite{2013-SIGMOD-TurboISO}, BoostIso~\cite{2015-VLDB-BoostIso}, and
DualIso~\cite{2014-IEEE-DualIso}.

A 2012 comparison of existing algorithms~\cite{2012-VLDB-IsoSurvey} concluded
that QuickSI, and GraphQL were the fastest from among other algorithms including
GADDI, SPath~\cite{2010-VLDB-SPath}, and VF2~\cite{2004-PAMI-VF2}. It created a
common framework for all of the algorithms, that allowed for a more
comprehensive understanding of the way in which these graphs are being
queried. It is essentially broken up into four steps. \textsc{FilterCandidates},
which performs a label search on a given edge.  Once this has been performed
for all vertices, the recursive subroutine \textsc{SubgraphSearch} is
called. Within this routine, there is the function \textsc{NextQueryEdge},
which determines the search order of the query graph, \textsc{IsJoinable}, which
determines whether the proposed match is actually viable, \textsc{UpdateState},
which updates the mapping with the joinable pair, then the recursive call, and
finally, \textsc{RestoreState}, which removes the pair from the mapping. This is
explicitly stated in Algorithm~\ref{alg:gen_query_proc}. Note that a key
difference between this algorithm and the one presented in
\cite{2012-VLDB-IsoSurvey} is that here we have an algorithm that finds an
edge-mapping as opposed to a vertex-mapping, and so the structure is slightly
different. 

\begin{algorithm}
  \label{alg:gen_query_proc}
  \caption{\textsc{GenericQueryProc}$(Q,G)$}
  \SetAlgoLined
  \KwIn{A query graph $Q$, A data graph $G$}
  \KwOut{All subgraph ismorphisms of $Q$ in $G$}

  Initialize the Mapping $M$ to $\emptyset$\;
  \ForEach{$e \in E(Q)$}{
    $\Phi(e)$ \texttt{:=} \textsc{FilterCandidates}$(G,Q,e, \cdots )$\;
    \If{$\Phi(e) = \emptyset$}{ \Return{} \; }
  }

  \textsc{SubgraphSearch}$(Q,G,M,\Phi, \cdots)$\;

  \setcounter{AlgoLine}{0}
  \SetKwProg{subroutine}{Subroutine}{}{}
  \subroutine{\textsc{SubgraphSearch}$(Q,G,M,\Phi, \cdots)$}{
    \eIf{$|M| = |E(Q)|$}{
      \textbf{Report} $M$\;
    }{
      $e$ \texttt{:=} \textsc{NextQueryEdge} $(\cdots)$\;
      $\Phi'(e)$ \texttt{:=} \textsc{RefineCandidates} $(M,u, \Phi(e), \cdots)$\;
      \ForEach{$f \in \Phi'(e)$ that is not yet matched}{
        \If{\textsc{IsJoinable}$(Q,G,e,f, \cdots )$}{
          \textsc{UpdateState}$(M,e,f, \cdots )$\;
          \textsc{SubgraphSearch}$(Q,G,M, \cdots)$\;
          \textsc{RestoreState}$(M,e,f, \cdots )$\;
        }
      }
    }
  }
\end{algorithm}

In the next section we will detail how to go about developing this framework for
existing graphs. Specifically, how we can use temporal information to further
restrict the search space for existing algorithmic paradigms.

\subsection{Temporal Postcondition}
\label{sec:postcondition}

Similar to the way in which \cite{2016-arXiv-TemporalIso} developed several
naive versions of the VF2~\cite{2004-PAMI-VF2} algorithm to include the basics
of the \textsc{Wconsec} temporal semantics. We will consider a similar algorithm
to the \textit{To-Ti}, algorithm where the topographical information is
considered before the temporal information. Here, we simply filter the results
of any implementation of \textsc{GenericQueryProc} with $\tau_b \circ T \circ
E$. We get this naive algorithm in Algorithm~\ref{alg:naive_temp}.

\begin{algorithm}
  \label{alg:naive_temp}
  \caption{\textsc{TopTimeQuery}$(<Q,T_q>, G, c)$}
  \KwIn{A temporal query graph $Q$, a time range $T_q$, a
    data graph $G$, and a temporal condition $c$}
  \KwOut{The set of patterns obeying the temporal semantic $c$ matching $Q$ in $G$ }

  $R$ \texttt{:=} \textsc{GenericQueryProc}$(Q,G)$\;
  
  \ForEach{$g \in R$}{

    \eIf{$c \neq \emptyset$ and $T_q \neq \emptyset$}{
      \If{ not $\tau_b(c)(\{T(e) | e \in E(G)\} \cup \{T_q\})$}{
        $R$\texttt{.remove} g\;
        \textbf{next}\;
      }
    }{

      \If{$c \neq \emptyset$ and not $\tau_b(c)(\{T(e) | e \in E(G)\}$}{
        $R$\texttt{.remove} g\;
        \textbf{next}\;
      }
      
      \If{$T_q \neq \emptyset$ and not $\forall e \in
        E(g).\tau_B(\textsc{concur})\{T(e),T_q\}$}{ $R$\texttt{.remove} g\;
        \textbf{next}\; } } \ForEach{$e' \in V(g)$}{ Let $e \in V(Q)$ be the
      vertex mapped to $e'$\; \If{$T(e) \cap T(e') = \emptyset$}{
        $R$\texttt{.remove} g\; \textbf{break}\; } } } \Return $R$\;
\end{algorithm}

This algorithm simply filters out those result graphs that violate the temporal
semantics and/or the time window $T_q$.  When both $c$ and $T_q$ are given,
the potential result graph must take $T_q$ into consideration when checking if
the potential result graph obeys $c$.


\subsection{Preprocessing on Query Graph}

\subsubsection{Constraint Tightening}
Oftentimes a query graph will have contradictory or superfluous temporal
information, for example, the local interval of a given edge may not intersect
at all with the temporal condition or the local intervals given will preclude
the explicit semantics, in which case, we can reject the query instantly with a
null result.

Another interesting situation is when, for a given query $<Q, T_q,
\intersection, \concurrent>$\todo{change earlier definition to reflect this
  definition}, and some edge $e \in E(G)$ with $T(e) \neq T(e) \cap T_q \neq
\emptyset$. Then, if we enforce the \intersection{} semantics, we do not need to
consider, for the edge $e$, the part of the interval $T(e) - T_q$, so we can
update $T(e)$ to be $T(e) \cap T_q$.

Further, we need to consider the way that the \concurrent{} semantics will
propogate to the neighbors of $e$. For this case, we need to enforce the
\intersection{} semantics for every edge in $Q$, meaning that we can rewrite
each $T(e)$ to be $T_q \cap \bigcap_{e' \in E} T(e')$. Of course for the
$\exact$ semantics this loses pruning power, since we can just test perform an
index search on the edges of the data-graph to find all time-restricted
candidates. We can perform a similar tightening for almost every combination
in $\impTC \times \expTC$. Here are the ones for which a tightening makes sense,
note that after transformation, the explicit and implicit semantics remain the same.

\begin{center}
  \begin{tabular}{cc c c} \toprule
    $\impTC$      & $\expTC$   & start & end \\ \midrule
    \concurrent   & \contain, \intersect
      & $\min \left(\bigcap_{e' \in N(e)} T(e')\right)$ 
      & $\max \left(\bigcap_{e' \in N(e)} T(e') \right)$ \\
    \strongConsec & \contain, \intersection
      & $\min \left(\bigcup_{e' \in pred(e)} T(e')\right)$
      & $\max \left(\bigcup_{e' \in succ(e)} T(e')\right)$ \\
    \strongConsec & \contained
      & $\max \left(\bigcup_{e' \in pred(e)} T(e')\right)$
      & $\min \left(\bigcup_{e' \in succ(e)} T(e')\right)$ \\
    \bottomrule
  \end{tabular}
\end{center}

The above rewriting rules are only improvements on less restrictive intervals
and will not always be an improvement on the given rules (except in the case of
the \concurrent{} semantics. 

\todo[inline]{We need proofs of these.  In an actual paper these should go in an
Appendix, since for the purpose of the paper we don't care why they exist, just
that there are such rewrite rules.}

\subsubsection{Hypergraph Compression}

We can also identify substructures of the query graph for which the semantic
conditions are equivalent. What? that's possible? tell me more... Consider the
following example of a cycle. Its a well-known fact that interval graphs that
contain a cycle are chordal, so for 2-cycles all three semantics are equivalent,
and for 3-cycles the \concurrent{} and \strongConsec{} semantics are
equivalent.

We can also note that for a star in which the in- and out-degrees of the central
vertex are at least one, the \concurrent{} and \strongConsec{} semantics
are equivalent.

So, given a specific query, we can decompose it maximually into 2-cycles, and
3-cycles (i.e. if a 3-cycle contains a 2-cycle add the 3-cycle instead of the
2-cycle) and enforce the temporal constraints enforced by the tight
semantics. Note that these are not the only structures for which the semantics
are the same, but they were the easiest to detect.

\begin{conjecture}
  3-cycles are the largest structures (w.r.t. number of vertices) that allow us to
  reduce \strongConsec{} semantics to \concurrent{}.
\end{conjecture}

\begin{proof}
  \todo[inline]{fill in here}.
\end{proof}

So, we will detect such two and three cycles using an $O(|V_Q||E_Q|)$ method,
since we are assuming small query graphs with no more than 20 or 30 edges, we
can store this in memory. The detection algorithm is defined in
\ref{alg:detect_substructs}. It only makes sense to run this algorithm for
$\weakConsec$ and $\strongConsec$ since we are attempting to leverage the
selectivity of the $\concurrent$ semantics.

\begin{algorithm}
  \label{alg:detect_substructs}
  \caption{\textsc{DetectSubstructs}}
  \KwIn{A query graph $Q$ and an implicit semantic $m$}
  \KwOut{A set of structures reducing to $\concurrent$}
  \SetAlgoLined

  Let the set of cycles $C$ \texttt{:=} $\emptyset$\;
  \ForEach{$e = (u,v) \in E(Q)$}{
    \ForEach{$e' = (v,u) \in E(Q)$}{
      add $\{e,e'\}$ to $T$\;
    }
    \If{$m \neq \weakConsec$}{
      \ForEach{$w \in V(Q)$}{
        \ForEach{ pair of edges $e' = (v, w), e'' = (w,u)$}{
          add $(e,e', e'')$ to $T$\;
        }
      }
    }
  }
\end{algorithm}

Then, for each of these substructures $S_e$, we will build a hypernode, given $m
\in \impTC$, and $x \in \expTC$, $N_{S_e}^{m}$ that reduces to $\concurrent$
semantics. This hypernode will then have an active window itself.

We will then define a hypernode profile, which is an extended version of the
temporal profile (see section \ref{sec:encoding}). For a given node $n$, it is a
tuple of the lexographically ordered labels ($p_s$),
$\impApprox(\concurrent)(S_e)$, and $\impApprox(m)(S_e)$. When we traverse the
tree to try and map these vertices, we will need to prune results such that the
internal edges obey the $\intersection$ explicit semantics with
$\impApprox(\concurrent)(S_e)$, and the incoming and outgoing edges obey the $m$
implicit semantics with respect to $\impApprox(m)(S_e)$. Algorithm
\ref{alg:hypernode_matching} describes the hypernode matching process.

\begin{algorithm}
  \label{alg:hypernode_matching}
  \caption{HypernodeMatching}

  \KwIn{A query hypernode $n$ and its profile $(p_s, p_{in}, p_{out})$, a
    data hypernode $n'$ and its profile $(p'_s, p'_{in}, p'_{out})$, the
    set of candidate sets $\Phi$}

  \KwOut{The updated candidates sets $\Phi$}
  \SetAlgoLined
  

  \If{$p_s \neq p'_s$ or $p_{in} \cap p'_{in} = \emptyset)$ or $|E(n)| > |E(n')|$}{
    \ForEach{$e \in E(n)$}{
      remove $E(n')$ from $\Phi(e)$\;
    }
  }
\end{algorithm}

Once we have constructed such hypernodes, we can re-tighten the constraints on
the graph as per the rules defined above.

Further, when traversing the graph, we will enforce, for $f$ an incoming or
outgoing edge to a matched hypernode in the data graph, $\impApprox(m)\{T(e),
p_{out}\}$.

\subsection{Simple Modification of \textsc{FilterCandidates}}
\label{sec:naive_filter_candidates}

The method presented in section \ref{sec:postcondition} is very simplistic and
will result in a lot of unnecessary computation of branches of the search tree
that might've been pruned earlier. In this section we begin to propose some
simple modifications to existing algorithms that will combine to create a first
algorithm for temporal pattern matching. The extension discussed in
Section~\ref{sec:naive_joinable} is the one described in the temporal extension
of VF2 \cite{2016-arXiv-TemporalIso}.

The purpose of the function \textsc{FilterCandidates} is to provide a
label-based index-boosted search of potential candidates for the input edge
$e$. Some algorithms \ref{1976-ACMJ-Ullman, 2009-EDBT-GADDI, 2008-VLDB-QuickSI}
will only perform a label search. while others will perform some signature based
pruning \ref{2010-VLDB-SPath,2008-SIGMOD-GraphQL}, and still others perform
transformations on the query and data graphs \ref{2013-SIGMOD-TurboISO, 2015-VLDB-BoostIso}. We can
extend the basic label search to include some basic temporal information. If for
an edge $e$ in the query graph and a potential candidate $f$ in the data graph,
if there is some temporal information $T(e)$, then we will enforce
$\tau_b(c)(\{T(e), T(f)\})$. In the GraphQL algorithm, a function that discounts
candidates based on signatures is composed with the standard label-index. We
discuss our version of this algorithm in Section~\ref{sec:encoding}. The
advantage of doing this here, is that we don't need to consider the specific
temporal information of the edges in the query graph in the inner loops of
\textsc{SubgraphSearch} since we already know that they are matched
appropriately.


\subsection{Simple temporal extension of \textsc{RefineCandidates}}

\todo[inline]{figure out what to do here}

\subsection{Temporal modification of \textsc{NextQueryEdge}}

Here, we want to minimize the search space as early a possible and maintain the
smallest number of search options as long as possible. To pick a start vertex,
we will simply pick the edge $e \in E_Q$ that has the smallest $|\phi(e)|$ As in
many existing algorithms, \cite{2004-PAMI-VF2, 2009-EDBT-GADDI,
  2008-SIGMOD-GraphQL} we will limit our search to edges that are coincident to
already mapped edges, but have not been paired. Of course we will also limit to
those edges whose addition does not violate $\impBool(m)$ Then we need to
specify an optimal order among these.  We will select the edge whose addition
minimizes the result of $\impApprox(m)$, run on those edges matched so far, and
the new edges to be added.

\textbf{Remarks}
We can also consider other filtering methods.  Most of them can just be composed
with our existing ordering, probably after so that ties will be broken by the
following conditions.
\begin{itemize}
  \item Greedily minimize size of intermediate results \cite{2008-SIGMOD-GraphQL} 
  \item Take a statistical profile of all edges in the query graph. Use the
    frequency as a weight. Obtain a minimum spanning tree (of the coincidence
    graph) and search in order of insertion into min spanning
    tree. \cite{2008-VLDB-QuickSI}
\end{itemize}

\subsection{Simple Modification of $\textsc{IsJoinable}$}
\label{sec:naive_joinable}

Now we want to similarly consider a more general form of the extension to VF2
\cite{2004-PAMI-VF2} done in \cite{2016-arXiv-TemporalIso}, in which the authors
somewhat informally presented a $Ti\&To$ algorithm in which they considered the
the temporal information as they considered the Topographical information by
extending the semantic function built into VF2. This is the \textsc{IsJoinable}
subroutine introduced in Algorithm~\ref{alg:gen_query_proc} following the
convention established in \cite{2012-VLDB-IsoSurvey}. To mirror this simple
extension, we will enforce the condition $\tau_B$ for every new edge introduced,
and reject the edge if one of them fails. Essentially, if a query vertex $u'$ is
adjacent to $u$ and has already been matched, then it ensures that there is a
corresponding edge in the data graph (with matching label if necessary). In
\cite{2004-PAMI-VF2}, they maintain the dates of previously accessed nodes to
assure that the current node maintains the \textsc{Wconsec} condition. However,
since we are finding a mapping between edges (which contains a mapping between
nodes), our mapping contains all of the edges that have been used, so we already
have the relevant information.

This algorithm is presented in Algorithm~\ref{alg:naive_isjoinable}. First, we
will need some notation. Let $M_Q$ be the domain-so-far, and let $M_G$ be the
image-so-far. I.e. $M_Q := \texttt{map fst} M$ and $M_G := \texttt{map snd} M$.
The algorithm relies on the invariant that the mapping-so-far is contemporary
with respect to some condition $c$.

\begin{algorithm}
  \label{alg:naive_isjoinable}
  \caption{\textsc{IsJoinable}$(Q,T_q,G,e,f,M,c)$}
  \KwIn{A query graph $Q$, $T_q$ a time interval, a data graph $G$, $e \in V(Q)$, $f \in G(Q)$, $c$ a
    contemporaneity condition, and $M \in P(E(Q)\times G(Q))$ the mapping so
    far}
  \KwOut{A boolean representing whether we can safely add the pair $e \mapsto f$
    to $M$}
  
  Let $e \texttt{:=} (u,u')$ and $f \texttt{:=} (v,v')$\;
  
  \ForEach{$e' \in (Pred(e) \cup Succ(e)) \cap M_Q$ }{

    \eIf{$\exists f' \in (Pred(e) \cup Succ(e)) \cap M_G. (e \mapsto f') \in M$}{
      \eIf{$c \neq \emptyset$ and $T_q \neq \emptyset$ }{
        \If{not $\tau(c)(\{T_q\} \cup \{T(f'') | f'' \in M_G \cup \{f\}\})$}{
          \Return \emph{False}\;
        }
      }{
        \If {$c \neq \emptyset$ and not $\tau(c)(\{T(f'') | f'' \in M_G \cup \{f\}\})$}{
          \Return \emph {False}\;
        }
        \If{$T_q \neq \emptyset$ and not $\tau(\texttt{consec})(\{T(f),T_q\})$}{  
          \Return \emph{False}\;
        }
      }
    }{
      \Return \emph{False}\;
    }
    
    \Return \emph{True}\;
  }
\end{algorithm}


This is obviously going to significantly reduce the search space from the naive
approach presented in section \ref{sec:postcondition}. This will prune very
early the execution large sections of the tree that will not be searchable since
they rely on a non-consecutive or non-intersecting edge.

