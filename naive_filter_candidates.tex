\subsection{Simple Modification of \textsc{FilterCandidates}}
\label{sec:naive_filter_candidates}

The method presented in section \ref{sec:postcondition} is very simplistic and
will result in a lot of unnecessary computation of branches of the search tree
that might've been pruned earlier. In this section we begin to propose some
simple modifications to existing algorithms that will combine to create a first
algorithm for temporal pattern matching. The extension discussed in
Section~\ref{sec:naive_joinable} is the one described in the temporal extension
of VF2 \cite{2016-arXiv-TemporalIso}.

The purpose of the function \textsc{FilterCandidates} is to provide a
label-based index-boosted search of potential candidates for the input edge
$e$. Some algorithms \ref{1976-ACMJ-Ullman, 2009-EDBT-GADDI, 2008-VLDB-QuickSI}
will only perform a label search. while others will perform some signature based
pruning \ref{2010-VLDB-SPath,2008-SIGMOD-GraphQL}, and still others perform
transformations on the query and data graphs \ref{2013-SIGMOD-TurboISO, 2015-VLDB-BoostIso}. We can
extend the basic label search to include some basic temporal information. If for
an edge $e$ in the query graph and a potential candidate $f$ in the data graph,
if there is some temporal information $T(e)$, then we will enforce
$\tau_b(c)(\{T(e), T(f)\})$. In the GraphQL algorithm, a function that discounts
candidates based on signatures is composed with the standard label-index. We
discuss our version of this algorithm in Section~\ref{sec:encoding}. The
advantage of doing this here, is that we don't need to consider the specific
temporal information of the edges in the query graph in the inner loops of
\textsc{SubgraphSearch} since we already know that they are matched
appropriately.

