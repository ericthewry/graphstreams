\section{Problem Definition}
\label{sec:definitions}

In this section, we provide a set of formal definitions of temporal graph and
temporal graph query. Then, we will discussion different semantics for
interpreting the graph pattern matching problem on temporal graphs.

A temporal graph is a graph that is annotated with time (of domain
$\Z$). Formally,

\begin{defn}
  A \textbf{temporal graph} is a node and edge labeled $G = (V_G,E_G)$ where
  $V_G$ is the set of vertices, and $E_G \subset V^2\times \Z^2$ is the set of
  edges. A label function $L_G$ maps each node and each edge to its label.
  
  Associated with each edge $e$ is  a time interval $(ts_e, tf_e)$, which 
  we call the {\bf active period} of the edge. 
\end{defn}

  We provide a few helper functions to obtain the end nodes of each edge and the
  set the outgoing and incoming edges of a node:
  \begin{itemize}
  	\item given a node $u \in V_G$, $out(u) = \{e \in E_G \ | \,e = (u,v)$
          for some $v \in V_G\}$; $in(u) = \{e \in E_G \,|\, e = (v,u)$ for some
          $v \in V_G\}$.
	\item given an edge $e = (u,v) \in E_G$, $\pi_1(e) = u$ and $\pi_2(e) =
          v$.
  \end{itemize}
  
   $\T : E \to \Z^2$ is a function that for any $e \in E_G$, $\T(e) = (ts_e,
  tf_e)$.

  We provide two helper function $\T_s$ and $\T_f$ that return the start and
  finish time of the active period of an edge: $\T_s(e) = ts_e$ and $\T_f(e) =
  tf_e$.
  
  Given two time intervals $T_1=(ts_1, tf_1)$ and $T_2 =(ts_2, tf_2)$, we define
  the following predicates and computation:
  \begin{itemize}
  	\item  $T_1 = T_2 \Leftrightarrow ts_1 = ts_2 \wedge tf_1 = tf_2$;
	\item $T_1 \subseteq T_2 \Leftrightarrow ts_1 \ge ts_2 \wedge tf_1 \le tf_2$;  
 	\item $T_1 \cap T_2 = (max(ts_1, ts_2), min(tf_1, tf_2))$;
\end{itemize}

 A time interval $ (ts, tf) = \emptyset$, if $tf<ts$.
\SmallSpace

\noindent{\bf Remarks:} 

\begin{enumerate}
	\item $G$ is not necessarily connected. 
	
	\item There maybe more than one edge between a pair of nodes, bearing
          different active period $(t_s, t_f)$.
		
	\item For each pair of edges $e_1$ and $e_2$ between the same pair of
          nodes $(u,v)$ that have the same edge label, we can assume that the
          active period of the edges do not overlap, e.g., $\T(e_1) \cap \T(e_2)
          = \emptyset$, since if they do overlap, we can combine the two edges
          to form one whose active period is the union of the two.

	\item There can be many simplified versions of the node and edge
          labeling. For instance, only nodes are labeled, but edges are not,
          hinting that all edges are labeled the same.
	
	\item In this definition, we only associate edges with timestamps. We
          can assume that nodes are always active.
	
	\item In this definition, $G$ is a directed graph.  To make it
          undirected, we can
	\begin{itemize}
		\item define $E \subseteq P^2(V) \times \Z^2$, where $P^2(V)$ is
                  the powerset of size two over the naturals.
		\item require that $(u,v,t_s,t_f) \in E_G \rightarrow
                  (v,u,t_s,t_f) \in E_G$
	\end{itemize}
\end{enumerate}

We define graph patterns in a way similar to how graph patterns are defined in
SPARQL-like queries, but allowing users to provide additional constraints on
time.

\begin{defn}
  A {\bf temporal graph query $q = <G_q, T_q>$} consists of a graph pattern
  $G_q$ and a time interval $T_q$.

Both the start and finish time of $\T_q$ can be a constant or ?.

The graph pattern is a connected graph $G_q = (V_q, E_q)$, where $V_q$ is the
set of nodes and $E_q$ is the set of edges. A label function $L_q$ maps each
node/edge to its label, which can also be ?. Associated with each edge in $E_q$,
user can also provide a temporal constraint in the form of a time interval,
again, both the start and finish time can be a constant or ?.

We call $T_q$ the {\bf global temporal constraint} of $q$ and $T(e_q)$ for each
$e_q \in E_q$ the {\bf local temporal constraints}.
\end{defn}

We overload the helper functions introduced earlier to apply to graph pattern
and time intervals that serve as temporal constraints.

\noindent {\bf Remark:} the definition above can be incorporated easily into
SPARQL. We can investigate the details when we settle on the definition.

\begin{defn}
  \label{defn:match}
  A {\bf match} of a graph pattern $G_q$ in $G$ is a total mapping $h: \{e_q: e_q
  \in E_q\} \rightarrow \{e_G: e_G \in E_G\} $ such that:

  \begin{itemize}
    \item for each edge $e_q \in E_q$, the edge label predicate associated with
      $e_q$ is satisfied by $h(e_q) \in E_G$.
    \item for each node $v_q \in V_q$, the mapping of the outgoing and incoming
      edges of $v_q$ share the same end node $v_G \in V_G$ and the node label
      predicate associated with $v_q$ is satisfied by $v_G$. Formally, for any
      two edges $e_1, e_2 \in E_q$, if $\pi_i(e_1) = \pi_j(e_2)$, where , $i ,j$
      can be 0 or 1, the following must hold: $\pi_i(h(e_1)) = \pi_j(h(e_2))$.
\end{itemize}
\end{defn}

Please note that the definition of matching is the same as pattern matching
defined for SPARQL-like graph queries. The new problem is how we can take the
temporal constraints into consideration, which will be defined next.

Note that we allow users to provide temporal constraints in the form of a time
window for the whole pattern and for each edge, but we also provide the
flexibility for them not to provide any specific temporal constraints via the
``?'' option. Hence, users' temporal specification can be very strict, or very
relaxed, or anywhere in between. Here are some scenarios:

\begin{itemize}
        \item most strict: user specifies explicit global and local temporal
          constraints, and for each constraint specified, the start time is the
          same as the end time.
	\item most relaxed: user specifies temporal constraints with all ?'s, which means infinity.
	\item anywhere in between: including the cases in which some temporal constraints contains ?. 
        \item conflicted: the intersection of the global temporal constraint and
          at least one of the local temporal constraint is empty. {\bf remark:}
          we can easily identify conflict cases and return empty results without
          query evaluation.
\end{itemize}

We first define a few semantics that explicitly address user-specified temporal
constraints:

\begin{defn}
\label{explicit-temp-semantics}
Given a graph $G$, a temporal graph query $q= <G_q, T_q>$, we say that a graph
pattern matching $h$ explicitly satisfies the temporal constraint of $q$ if
\begin{itemize}
	\item under the {\bf exact} semantics, $h$ satisfies that:
	\begin{itemize}
		\item for all $e_q \in E_q$, $T(h(e_q)) = T(e_q)$ and $T(h(e_q))
                  \subseteq T_q$.
	\end{itemize}

	\item under the {\bf contain} semantics, $h$ satisfies that:
	\begin{itemize}
		\item for all $e_q \in E_q$, $T(h(e_q)) \subseteq T(e_q)$ and
                  $T(h(e_q)) \subseteq T_q$.
	\end{itemize}

	\item under the {\bf contained} semantics, $h$ must satisfy that:
	\begin{itemize}
		\item for all $e_q \in E_q$, $T(h(e_q)) \supseteq T(e_q)$ and
                  $T(h(e_q)) \supseteq T_q$.
	\end{itemize}

	\item under the {\bf intersection} semantics, $h$ must satisfy that:
	\begin{itemize}
		\item for all $e_q \in E_q$, $T(h(e_q)) \cap T(e_q) \not=
                  \emptyset $ and $T(h(e_q)) \cap T_q \not= \emptyset$.
	\end{itemize}

\end{itemize}
\end{defn}

\noindent{\bf Remark:} We need to think more carefully about how global temporal
constraint $T_q$ is interpreted in these semantics. Best way is to come up with
some example queries.

\SmallSpace

The {\bf {\em explicit temporal semantics}} defined above address only the issue
of interpreting temporal constraints specified by users. Matching returned under
all these semantics will include subgraphs that are not temporally traversable.

We next define a few {\bf {\em implicit temporal semantics}} as remedy. 

\begin{defn}
  \label{def:concur}
  Given a temporal graph $G$, we say that the graph is {\bf concurrent} if
  $\displaystyle\bigcap_{e \in E_G}T(e) \neq \emptyset$.
\end{defn}

\begin{defn}
  \label{def:consec}  
  Given a temporal graph $G$, we say that the graph is \textbf{weakly
    consecutive} if for every $e = (w,u,t_s, t_f), e' = (u,v,t_s', t_f') \in
  E_G$, $t_f' \geq t_s$, where we say it is {\bf strongly consecutive} if for
  any $e = (w,u), e'=(u,v) \in E_G$, $T(e) \cap T(e') \neq \emptyset$.
\end{defn}

Hence, we can define {\bf {\em implicit temporal semantics}} to demand that
resultant matching sub-graph be \textbf{concurrent}, \textbf{weakly
  consecutive}, or \textbf{strongly consecutive}.

\SmallSpace
\noindent {\bf Conjectures:}
 \begin{enumerate}
	\item If the graph $G$ is of star shape, e.g., there exist a center node
          $u$ such that for any other nodes $v$ there exists an edge between $u$
          and $v$, then, $G$ is concurrent iff $G$ is consecutive.
	\item If the graph $G$ is a clique, then, $G$ is concurrent iff $G$ is
          consecutive.2
	\item If there exists a circle that traverse through all nodes in $G$,
          then, there exist an edge on this circle whose active period dominates
          those of all other edges on the circle.
	
\end{enumerate}

\noindent {\bf Remarks:}
\begin{itemize}
	\item Whether these conjectures are true or not may depend on whether
          the graph is directed or undirected.
	\item Either we need to find some work that has proved them, or to prove
          them, which should not be hard.
	\item These conjectures, if proved true, can be used to speed up query
          evaluation process. It should be easy to identify star shape, clique
          and circle in query patterns, which are usually very small. This will
          allow us to strength the temporal constraints based on the properties
          described in the conjectures to enhance filtering.

\end{itemize}
