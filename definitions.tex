\section{Problem Definition}
\label{sec:definitions}

In this section, we provide a set of formal definitions of temporal graph and
temporal graph query. Then, we will discussion different semantics for
interpreting the graph pattern matching problem on temporal graphs.

\subsection{Temporal Dimension}
Given two time intervals $T_1,T_2  \in \N^2$, we define the following predicates and computation:
\begin{itemize}
	\item $\pi_1(T_1) =\ts_1$, and $\pi_2(T_1) = \tf_1$;
  	\item $T_1 = T_2 \Leftrightarrow \ts_1 = \ts_2 \wedge \tf_1 = \tf_2$;
	\item $T_1 \subseteq T_2 \Leftrightarrow \ts_1 \ge \ts_2 \wedge \tf_1 \le \tf_2$;  
 	\item $T_1 \cap T_2 = (max(\ts_1, \ts_2), min(\tf_1, \tf_2))$;
	\item $T_1 \cup T_2 = (min(\ts_1, \ts_2), max(\tf_1, \tf_2))$;
\end{itemize}

\subsection{Temporal Graph}
A temporal graph is a node and edge-labeled graph that is annotated with time
intervals. Formally,

\begin{defn}
  A \textbf{temporal graph} is a node and edge labeled $G = (V_G, E_G, L_G)$
  where $V_G$ is the set of vertices, and $E_G \subset V_G^2 \times \N^2$ is the
  set of edges, and $L_G: V_G \cup E_G \rightarrow \Lb$ is a label function that
  maps each node and each edge to a label in domain $\Lb$.
  
 We call the time interval $(\ts, \tf)$ associated with each edge $e$ the {\bf
   active period} of $e$.
\end{defn}

  We provide a few helper functions to obtain the end nodes of each edge and the
  set the outgoing and incoming edges of a node:
  \begin{itemize}
    \item given a node $u \in V_G$, $out(u) = \{e \in E_G \ | \,e = (u,v)$ for
      some $v \in V_G\}$; $in(u) = \{e \in E_G \,|\, e = (v,u)$ for some $v \in
      V_G\}$.
    \item given an edge $e = (u,v) \in E_G$, $\pi_1(e) = u$ and $\pi_2(e) = v$.
  \end{itemize}
  
$\T : E \to N^2$ is a function retrieve the active period for an edge $e \in
  E_G$, $\T(e) = (\ts_e, \tf_e)$.

We provide two helper function $\T_s$ and $\T_f$ that return the start and
finish time of the active period of an edge: $\T_s(e) = \pi_1(T(e))$ and
$\T_f(e) = \pi_2(T_(e))$.
  

\SmallSpace

\noindent{\bf Remarks:} 

\begin{enumerate}
	\item $G$ is not necessarily connected.
	
	\item There maybe more than one edge between a pair of nodes, bearing
          different active period $(\ts, \tf)$.
		
	\item For each pair of edges $e_1$ and $e_2$ between the same pair of
          nodes $(u,v)$ that have the same edge label, we can assume that the
          active period of the edges do not overlap, e.g., $\T(e_1) \cap \T(e_2)
          = \emptyset$, since if they do overlap, we can combine the two edges
          to form one whose active period is $\T(e_1) \cup \T(e_2)$.

	\item There can be many simplified versions of the node and edge
          labeling. For instance, only nodes are labeled, but edges are not,
          hinting that all edges are labeled the same.
	
	\item In this definition, we only associate edges with timestamps. We
          can assume that nodes are always active.
	
	\item In this definition, $G$ is a directed graph.  To make it
          undirected, we can
	\begin{itemize}
		\item define $E \subseteq P^2(V) \times \N^2$, where $P^2(V)$ is
                  the powerset of size two over the naturals.
		\item require that $(u,v, \ts, \tf ) \in E_G \rightarrow
                  (v,u,\ts, \tf) \in E_G$
	\end{itemize}
\end{enumerate}

\begin{defn}
Given a graph $G = (V_G, E_G, L_G)$ and a set of edges $S_e \subseteq E_G$, we
say that $\pi_{S_e}(G) = (V_{S_e}, S_e, L_G)$, where $V_{S_e} = \bigcup_{e \in
  S_e} \{\pi_1(e)\} \cup \{\pi_2(e)\}$, is the edge-induced graph of $S_e$ on
$G$.
\end{defn}

\subsection{Temporal Graph Pattern}
We define graph patterns in a way similar to how graph patterns are defined in
sub-graph isomorphism problems, but allowing users to provide additional
constraints on time.

\begin{defn}
  A {\bf temporal graph query $q = <G_q, T_q>$} consists of a graph pattern
  $G_q$ and a time interval $T_q$.

$\T_q \in (\N \cup \{?\})^2$ is called the {\bf global temporal constraint} of
  $q$.

The graph pattern is a connected graph $G_q = (V_q, E_q, L_q)$, where $V_q$ is
the set of nodes and $E_q \in V_q^2 \times (\N \cup \{?\})^2$ is the set of
edges, and $L_q: V_q \cup E_q \rightarrow \Lb \cup \{?\}$ is a label function
that maps each node/edge to a label in $\L$ or ?.

Associated with each edge in $E_q$, user can also provide a temporal constraint
in the form of a time interval, again, both the start and finish time can be a
constant or ?.  We call $T(e_q)$ for each $e_q \in E_q$ the {\bf local temporal
  constraints}.
\end{defn}

We overload the helper functions introduced earlier to apply to graph pattern
and time intervals that serve as temporal constraints. In the computations and
operations associated with time intervals, $?$ is interpreted as $-\infty$ when
used as start time, and $\infty$ when used as finish time.

\noindent {\bf Remark:} the definition above can be incorporated easily into
SPARQL. We can investigate the details when we settle on the definition.

\subsection{Temporal Graph Pattern Matching}

\begin{defn}
  \label{defn:match}
  A {\bf match} of a graph pattern $G_q$ in $G$ is a total mapping $h: \{e_q:
  e_q \in E_q\} \rightarrow \{e_G: e_G \in E_G\} $ such that:

  \begin{itemize}
    \item for each edge $e_q \in E_q$, the edge label predicate associated with
      $e_q$ is satisfied by $h(e_q) \in E_G$.
    \item for each node $v_q \in V_q$, the mapping of the outgoing and incoming
      edges of $v_q$ share the same end node $v_G \in V_G$ and the node label
      predicate associated with $v_q$ is satisfied by $v_G$. Formally, for any
      two edges $e_1, e_2 \in E_q$, if $\pi_i(e_1) = \pi_j(e_2)$, where $i ,j$
      can be 0 or 1, the following must hold: $\pi_i(h(e_1)) = \pi_j(h(e_2))$.
\end{itemize}
\end{defn}

Please note that the definition of matching is the same as pattern matching
defined for conjunctive queries on graphs. The new problem is how we can take
the temporal constraints into consideration, which will be defined next.

Note that we allow users to provide temporal constraints in the form of a time
window for the whole pattern and for each edge, but we also provide the
flexibility for them not to provide any specific temporal constraints via the
``?'' option. Hence, users' temporal specification can be very strict, or very
relaxed, or anywhere in between. Here are some scenarios:

\begin{itemize}
        \item most strict: user specifies explicit global and local temporal
          constraints, and for each constraint specified, the start time is the
          same as the end time.
	\item most relaxed: user specifies temporal constraints with all ?'s,
          which means infinity.
	\item anywhere in between: including the cases in which some temporal
          constraints contains ?.
        \item conflicted: the intersection of the global temporal constraint and
          at least one of the local temporal constraint is empty. {\bf remark:}
          we can easily identify conflict cases and return empty results without
          query evaluation.
\end{itemize}

We first define a few semantics that explicitly address user-specified temporal
constraints:

\begin{defn}
\label{explicit-temp-semantics}
Given a graph $G$, a temporal graph query $q= <G_q, T_q>$, we say that a graph
pattern matching $h$ explicitly satisfies the temporal constraint of $q$ if
\begin{itemize}
	\item under the \exact{} semantics, $h$ satisfies that:
	\begin{itemize}
		\item for all $e_q \in E_q$, $T(h(e_q)) = T(e_q)$ and $T(h(e_q))
                  \subseteq T_q$.
	\end{itemize}

	\item under the \contain{} semantics, $h$ satisfies that:
	\begin{itemize}
		\item for all $e_q \in E_q$, $T(h(e_q)) \subseteq T(e_q)$ and
                  $T(h(e_q)) \subseteq T_q$.
	\end{itemize}

	\item under the \contained{} semantics, $h$ must satisfy that:
	\begin{itemize}
		\item for all $e_q \in E_q$, $T(h(e_q)) \supseteq T(e_q)$ and
                  $T(h(e_q)) \supseteq T_q$.
	\end{itemize}

	\item under the \intersection{} semantics, $h$ must satisfy that:
	\begin{itemize}
		\item for all $e_q \in E_q$, $T(h(e_q)) \cap T(e_q) \not=
                  \emptyset $ and $T(h(e_q)) \cap T_q \not= \emptyset$.
	\end{itemize}
\end{itemize}



  We will define $\expTC = \{\exact, \contain, \contained, \intersection\}$ to
  be the set of differentiating identifiers for the explicit semantics, and
  $\expVar$ to be one of them.

  Finally, we will define a curried function $\expLocal : \expTC \to P(E^2) \to
  \textbf{2}$ which will take an explicit temporal condition return a function
  that takes a set of pairs of time intervals and determine whether each pair of
  time intervals satisfies the given explicit temporal semantics. Thus, for
  $\expVar \in \expTC$, $\expLocal(\expVar) : \E^2 \to \textbf{2}$ is a function
  that takes a set of pairs of edges and determines whether such a mapping is
  obeys the explicit semantics defined by $\expVar$.

\end{defn}

\noindent{\bf Remark:} We need to think more carefully about how global temporal
constraint $T_q$ is interpreted in these semantics. Best way is to come up with
some example queries.

\SmallSpace

The {\bf {\em explicit temporal semantics}} defined above address only the issue
of interpreting temporal constraints specified by users. Matching returned under
all these semantics will include subgraphs that are not temporally traversable.

We next define a few {\bf {\em implicit temporal semantics}} as remedy. 

\begin{defn}
  \label{def:concur}
  Given a temporal graph $G$, we say that the graph is {\bf concurrent} (i.e. it
  satisfies the \concurrent{} implicit semantics) if $\displaystyle\bigcap_{e
    \in E_G} T(e) \neq \emptyset$.
\end{defn}

\begin{defn}
  \label{def:consec}  
  Given a temporal graph $G$, we say that the graph is \textbf{weakly
    consecutive} (satisfying the \weakConsec{} implicit semantics) if for every
  $e_1 = (w,u, \ts_1 , \tf_1), e_2 = (u,v, \ts_2, \tf_2 \in E_G$, $\tf_2 \geq
  \ts_1$, where we say it is {\bf strongly consecutive} (w.r.t the
  \strongConsec{} implicit semantics) if for any $e_1 = (w,u), e_2=(u,v) \in
  E_G$, $T(e_1) \cap T(e_2) \neq \emptyset$.
\end{defn}

\begin{lemma}
If a temporal graph $G$ is strongly consecutive, it must be weakly consecutive.
\end{lemma}
\begin{proof}
  This is fairly obvious by Definition~\ref{def:consec}.
\end{proof}

We use $\impTC$ to represent the set of implicit temporal constraints. $\impTC =
\{\concurrent, \strongConsec, \weakConsec\}$.

Now, we can define {\bf {\em implicit temporal semantics}} to demand that
resultant matching sub-graph be \textbf{concurrent}, \textbf{weakly
  consecutive}, or \textbf{strongly consecutive}.

We now define a few functions that apply the implicit temporal semantics. 

\begin{defn}
  Given a temporal graph $G$, the function $\impBool$ is a curried Boolean
  function $\impBool : \impTC \to P(E) \to \textbf{2} $ that takes an implicit
  temporal constraint $\impVar \in \impTC$ and returns a function that takes a
  set of edges $S_e \subseteq E_G$ as input and returns true if the edge-induced
  graph $\pi_{S_e}(G)$ satisfies constraint $c$, and false otherwise. In other
  words for a given temporal condition $\impVar \in \impTC$, $\impBool(\impVar)
  : P(E) \to \textbf{2}$. For simplicity we will use $\impBool(\impVar)(S_e)$
  and $\impBool(c,S_e)$ for a given set $S_e \subseteq E$ interdependently.
\end{defn}

\noindent {\bf Conjectures:}
\begin{enumerate}
  \item Given a temporal graph $G$, a set of edges $S_e \subseteq E_G$ and a
    temporal constraint $\impVar \in \impTC$, $\impBool(\impVar)(S_e)
    \Rightarrow \forall S \subseteq S_e (\tau_B(c, S))$
\end{enumerate}
